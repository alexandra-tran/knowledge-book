%%%%%%%%%%%%%%%%%%%%%%%%%%%%%%%%%%%%%%%%%%%%%%%%%%%
%% LaTeX book template                           %%
%% Author:  Amber Jain (http://amberj.devio.us/) %%
%% License: ISC license                          %%
%%%%%%%%%%%%%%%%%%%%%%%%%%%%%%%%%%%%%%%%%%%%%%%%%%%

\documentclass[a4paper,11pt,oneside]{book}
\usepackage[T1]{fontenc}
\usepackage[utf8]{inputenc}
\usepackage{lmodern}
\usepackage{geometry}
% \usepackage{showframe}

\newgeometry{vmargin={15mm}, hmargin={12mm,17mm}}   % set the margins
\setlength {\marginparwidth }{2cm}

%%%%%%%%%%%%%%%%%%%%%%%%%%%%%%%%%%%%%%%%%%%%%%%%%%%%%%%%%
% Source: http://en.wikibooks.org/wiki/LaTeX/Hyperlinks %
%%%%%%%%%%%%%%%%%%%%%%%%%%%%%%%%%%%%%%%%%%%%%%%%%%%%%%%%%
\usepackage{hyperref}
\hypersetup{
colorlinks=true,citecolor=blue,linkcolor=blue,linktocpage=true
}
\usepackage{graphicx}
\usepackage[english]{babel}

%%%%%%%%%%%%%%%%%%%%%%%%%%%%%%%%%%%%%%%%%%%%%%%%%%%%%%%%%
% Formatting
%%%%%%%%%%%%%%%%%%%%%%%%%%%%%%%%%%%%%%%%%%%%%%%%%%%%%%%%%

\usepackage{knowledge}
\usepackage{todonotes}
\usepackage{xhfill}

%%%%%%%%%%%%%%%%%%%%%%%%%%%%%%%%%%%%%%%%%%%%%%%%%%%%%%%%%
% Math
%%%%%%%%%%%%%%%%%%%%%%%%%%%%%%%%%%%%%%%%%%%%%%%%%%%%%%%%%

\usepackage{amsfonts}


%%%%%%%%%%%%%%%%%%%%%%%%%%%%%%%%%%%%%%%%%%%%%%%%%%%%%%%%%%%%%%%%%%%%%%%%%%%%%%%%
% 'dedication' environment: To add a dedication paragraph at the start of book %
% Source: http://www.tug.org/pipermail/texhax/2010-June/015184.html            %
%%%%%%%%%%%%%%%%%%%%%%%%%%%%%%%%%%%%%%%%%%%%%%%%%%%%%%%%%%%%%%%%%%%%%%%%%%%%%%%%
\newenvironment{dedication}
{
   \cleardoublepage
   \thispagestyle{empty}
   \vspace*{\stretch{1}}
   \hfill\begin{minipage}[t]{0.66\textwidth}
   \raggedright
}
{
   \end{minipage}
   \vspace*{\stretch{3}}
   \clearpage
}

%%%%%%%%%%%%%%%%%%%%%%%%%%%%%%%%%%%%%%%%%%%%%%%%
% Chapter quote at the start of chapter        %
% Source: http://tex.stackexchange.com/a/53380 %
%%%%%%%%%%%%%%%%%%%%%%%%%%%%%%%%%%%%%%%%%%%%%%%%
\makeatletter
\renewcommand{\@chapapp}{}% Not necessary...
\newenvironment{chapquote}[2][2em]
  {\setlength{\@tempdima}{#1}%
   \def\chapquote@author{#2}%
   \parshape 1 \@tempdima \dimexpr\textwidth-2\@tempdima\relax%
   \itshape}
  {\par\normalfont\hfill--\ \chapquote@author\hspace*{\@tempdima}\par\bigskip}
\makeatother

%%%%%%%%%%%%%%%%%%%%%%%%%%%%%%%%%%%%%%%%%%%%%%%%%%%
% First page of book which contains 'stuff' like: %
%  - Book title, subtitle                         %
%  - Book author name                             %
%%%%%%%%%%%%%%%%%%%%%%%%%%%%%%%%%%%%%%%%%%%%%%%%%%%

% Book's title and subtitle
\title{\Huge \textbf{Stuff I Know}  \footnote{This is a footnote.} \\ \huge Sample book subtitle \footnote{This is yet another footnote.}}
% Author
\author{\textsc{Callum Cassidy-Nolan}\thanks{\url{cuppajoeman.com}}}


\begin{document}

\frontmatter
\maketitle

%%%%%%%%%%%%%%%%%%%%%%%%%%%%%%%%%%%%%%%%%%%%%%%%%%%%%%%%%%%%%%%
% Add a dedication paragraph to dedicate your book to someone %
%%%%%%%%%%%%%%%%%%%%%%%%%%%%%%%%%%%%%%%%%%%%%%%%%%%%%%%%%%%%%%%
\begin{dedication}
Dedicated to Calvin and Hobbes.
\end{dedication}

%%%%%%%%%%%%%%%%%%%%%%%%%%%%%%%%%%%%%%%%%%%%%%%%%%%%%%%%%%%%%%%%%%%%%%%%
% Auto-generated table of contents, list of figures and list of tables %
%%%%%%%%%%%%%%%%%%%%%%%%%%%%%%%%%%%%%%%%%%%%%%%%%%%%%%%%%%%%%%%%%%%%%%%%
\tableofcontents
\listoffigures
\listoftables
\listoftodos

\mainmatter

%%%%%%%%%%%
% Preface %
%%%%%%%%%%%
\chapter*{Preface}
This book contains things that I know, have proven or have learned.

\section*{Un-numbered sample section}
Lorem ipsum dolor sit amet, consectetur adipiscing elit. Duis risus ante, auctor et pulvinar non, posuere ac lacus. Praesent egestas nisi id metus rhoncus ac lobortis sem hendrerit. Etiam et sapien eget lectus interdum posuere sit amet ac urna. Aliquam pellentesque imperdiet erat, eget consectetur felis malesuada quis. Pellentesque sollicitudin, odio sed dapibus eleifend, magna sem luctus turpis.

\section*{Another sample section}
Lorem ipsum dolor sit amet, consectetur adipiscing elit. Duis risus ante, auctor et pulvinar non, posuere ac lacus. Praesent egestas nisi id metus rhoncus ac lobortis sem hendrerit. Etiam et sapien eget lectus interdum posuere sit amet ac urna. Aliquam pellentesque imperdiet erat, eget consectetur felis malesuada quis. Pellentesque sollicitudin, odio sed dapibus eleifend, magna sem luctus turpis, id aliquam felis dolor eu diam. Etiam ullamcorper, nunc a accumsan adipiscing, turpis odio bibendum erat, id convallis magna eros nec metus.

\section*{Structure of book}
% You might want to add short description about each chapter in this book.
The book is partitioned into different sections based on the area of mathematics it is involved with.

\section*{About the companion website}
The website\footnote{\url{https://github.com/cuppajoeman/stuff-I-know}} for this file contains:
\begin{itemize}
  \item A link to (freely downlodable) latest version of this document.
  \item Link to download LaTeX source for this document.
  \item Miscellaneous material (e.g. suggested readings etc).
\end{itemize}

%%%%%%%%%%%%%%%%%%%%%%%%%%%%%%%%%%%%
% Give credit where credit is due. %
% Say thanks!                      %
%%%%%%%%%%%%%%%%%%%%%%%%%%%%%%%%%%%%
\section*{Acknowledgements}
\begin{itemize}
    \item A special word of thanks to professors who wanted to make sure I understood and learned as much as possible Alfonso Gracia-Saz\footnote{\url{https://www.math.toronto.edu/cms/alfonso-memorial/}}, Jean-Baptiste Campesato\footnote{\url{https://math.univ-angers.fr/~campesato/}}, Z-Module, riv, PlanckWalk, franciman, qergle from \#math on \url{https://libera.chat/}.
\item I'm deeply indebted my parents, colleagues and friends for their support and encouragement.
\end{itemize}

%%%%%%%%%%%%%%%%
% NEW CHAPTER! %
%%%%%%%%%%%%%%%%
\chapter{Introductory Chapter}

\begin{chapquote}{Author's name, \textit{Source of this quote}}
``This is a quote and I don't know who said this.''
\end{chapquote}

\section{Section heading}
Lorem ipsum dolor sit amet, consectetur adipisicing elit, sed do eiusmod tempor incididunt ut labore et dolore magna aliqua. Ut enim ad minim veniam, quis nostrud exercitation ullamco laboris nisi ut aliquip ex ea commodo consequat. \\ Duis aute irure dolor in reprehenderit in voluptate velit esse cillum dolore eu fugiat nulla pariatur. Excepteur sint occaecat cupidatat non proident, sunt in culpa qui officia deserunt mollit anim id est laborum. \\ Lorem ipsum list:

1. Lorem ipsum dolor sit amet, consectetur adipiscing elit.

2. Duis ac mi magna, a consectetur elit.

3. Curabitur posuere erat \emph{dignissim ligula euismod} ut euismod nisi.

4. Fusce vulputate facilisis neque, et ornare mauris mattis vel.

5. Mauris sit amet nulla mi, vitae rutrum ante.

6. Maecenas quis nulla risus, vel tincidunt ligula.

7. Nullam ac enim neque, non \emph{dapibus} mauris.

8. Integer volutpat leo a orci suscipit eget rhoncus urna eleifend.

\noindent Lorem ipsum dolor sit amet, consectetur adipiscing elit. Duis risus ante, auctor et pulvinar non, posuere ac lacus. Praesent egestas nisi id metus rhoncus ac lobortis sem hendrerit. Etiam et sapien eget lectus interdum posuere sit amet ac urna\footnote{Lorem ipsum dolor sit amet, consectetur adipiscing elit. Duis risus ante, auctor et pulvinar non, posuere ac lacus.}:

\subsection{Lorem ipsum dolor sit amet, consectetur adipiscing elit.}
Lorem ipsum dolor sit amet, consectetur adipiscing elit. Duis risus ante, auctor et pulvinar non, posuere ac lacus. Praesent egestas nisi id metus rhoncus ac lobortis sem hendrerit. Etiam et sapien eget lectus interdum posuere sit amet ac urna. Aliquam pellentesque imperdiet erat, eget consectetur felis malesuada quis. Pellentesque sollicitudin, odio sed dapibus eleifend, magna sem luctus turpis, id aliquam felis dolor eu diam. Etiam ullamcorper, nunc a accumsan adipiscing, turpis odio bibendum erat, id convallis magna eros nec metus. Sed vel ligula justo, sit amet vestibulum dolor. Sed vitae augue sit amet magna ullamcorper suscipit. Quisque dictum ipsum a sapien egestas facilisis.

Lorem ipsum dolor sit amet, consectetur adipiscing elit. Duis risus ante, auctor et pulvinar non, posuere ac lacus. Praesent egestas nisi id metus rhoncus ac lobortis sem hendrerit. Etiam et sapien eget lectus interdum posuere sit amet ac urna.

\subsection{Lorem ipsum dolor sit amet, consectetur adipiscing.}
Lorem ipsum dolor sit amet, consectetur adipiscing elit. Duis risus ante, auctor et pulvinar non, posuere ac lacus. Praesent egestas nisi id metus rhoncus ac lobortis sem hendrerit. Etiam et sapien eget lectus interdum posuere sit amet ac urna. Aliquam pellentesque imperdiet erat, eget consectetur felis malesuada quis. Pellentesque sollicitudin, odio sed dapibus eleifend, magna sem luctus turpis, id aliquam felis dolor eu diam.

\subsection{Lorem ipsum dolor sit amet}
Lorem ipsum dolor sit amet, consectetur adipiscing elit. Duis risus ante, auctor et pulvinar non, posuere ac lacus. Praesent egestas nisi id metus rhoncus ac lobortis sem hendrerit. Etiam et sapien eget lectus interdum posuere sit amet ac urna. Aliquam pellentesque imperdiet\footnote{\url{www.example.com}} erat, eget consectetur felis malesuada quis. Pellentesque sollicitudin, odio sed dapibus eleifend, magna sem luctus turpis, id aliquam felis dolor eu diam. Etiam ullamcorper, nunc a accumsan adipiscing, turpis odio bibendum erat, id convallis magna eros nec metus. Sed vel ligula justo, sit amet vestibulum dolor. Sed vitae augue sit amet magna ullamcorper suscipit. Quisque dictum ipsum a sapien egestas facilisis.

In hac habitasse platea dictumst. Nullam turpis erat, porttitor ut pretium ac, condimentum sed dui. Praesent arcu elit, tristique sit amet viverra at, auctor quis tortor. Etiam eleifend posuere aliquam. Donec sed mattis sapien. Aenean urna arcu, suscipit at rutrum ac, adipiscing ac felis. Class aptent taciti sociosqu ad litora torquent per conubia nostra, per inceptos himenaeos. Praesent condimentum felis a ipsum ullamcorper semper. Vivamus eu odio sem, dictum luctus nunc. Etiam tincidunt venenatis dolor non pellentesque.

\subsection{Lorem ipsum dolor sit amet, auctor et pulvinar non}
Lorem ipsum dolor sit amet, consectetur adipiscing elit. Duis risus ante, auctor et pulvinar non, posuere ac lacus. Praesent egestas nisi id metus rhoncus ac lobortis sem hendrerit. Etiam et sapien eget lectus interdum posuere sit amet ac urna. Aliquam pellentesque imperdiet erat, eget consectetur felis malesuada quis. Pellentesque sollicitudin, odio sed dapibus eleifend, magna sem luctus turpis, id aliquam felis dolor eu diam. Etiam ullamcorper, nunc a accumsan adipiscing, turpis odio bibendum erat, id convallis magna eros nec metus. Sed vel ligula justo, sit amet vestibulum dolor. Sed vitae augue sit amet magna ullamcorper suscipit. Quisque dictum ipsum a sapien egestas facilisis.

In hac habitasse platea dictumst. Nullam turpis erat, porttitor ut pretium ac, condimentum sed dui. Praesent arcu elit, tristique sit amet viverra at, auctor quis tortor. Etiam eleifend posuere aliquam. Donec sed mattis sapien. Aenean urna arcu, suscipit at rutrum ac, adipiscing ac felis. Class aptent taciti sociosqu ad litora torquent per conubia nostra, per inceptos himenaeos. Praesent condimentum felis a ipsum ullamcorper semper. Vivamus eu odio sem, dictum luctus nunc. Etiam tincidunt venenatis dolor non pellentesque. 


\section{Another section heading}
Lorem ipsum dolor sit amet, consectetur adipisicing elit, sed do eiusmod tempor incididunt ut labore et dolore magna aliqua. Ut enim ad minim veniam, quis nostrud exercitation ullamco laboris nisi ut aliquip ex ea commodo consequat.

%%%%%%%%%%%%%%%%%%%%%%%%%%%%%%%%%%%%%%%%%%%%%%%%%%%%%%%
% Sample table                                        %
% Source: www1.maths.leeds.ac.uk/latex/TableHelp1.pdf %
%%%%%%%%%%%%%%%%%%%%%%%%%%%%%%%%%%%%%%%%%%%%%%%%%%%%%%%
\begin{table}[ht]
\caption{Sample table} % title of Table
\centering % used for centering table
\begin{tabular}{c c c c}
% centered columns (4 columns)
\hline\hline %inserts double horizontal lines
S. No. & Column\#1 & Column\#2 & Column\#3 \\ [0.5ex]
% inserts table
%heading
\hline % inserts single horizontal line
1 & 50 & 837 & 970 \\
2 & 47 & 877 & 230 \\
3 & 31 & 25 & 415 \\
4 & 35 & 144 & 2356 \\
5 & 45 & 300 & 556 \\ [1ex] % [1ex] adds vertical space
\hline %inserts single line
\end{tabular}
\label{table:nonlin} % is used to refer this table in the text
\end{table}

Duis aute irure dolor in reprehenderit in voluptate velit esse cillum dolore eu fugiat nulla pariatur. Excepteur sint occaecat cupidatat non proident, sunt in culpa qui officia deserunt mollit anim id est laborum. \\ Lorem ipsum list:
\begin{itemize}
\item Mauris sit amet nulla mi, vitae rutrum ante.
\item Maecenas quis nulla risus, vel tincidunt ligula.
\item Nullam ac enim neque, non \emph{dapibus} mauris.
\end{itemize}

\noindent Lorem ipsum dolor sit amet, consectetur adipiscing elit. Duis risus ante, auctor et pulvinar non, posuere ac lacus. Praesent egestas nisi id metus rhoncus ac lobortis sem hendrerit. Etiam et sapien eget lectus interdum posuere sit amet ac urna\footnote{Lorem ipsum dolor sit amet, consectetur adipiscing elit. Duis risus ante, auctor et pulvinar non, posuere ac lacus.}:

\subsection{Lorem ipsum dolor sit amet, consectetur adipiscing elit.}
Lorem ipsum dolor sit amet, consectetur adipiscing elit. Duis risus ante, auctor et pulvinar non, posuere ac lacus. Praesent egestas nisi id metus rhoncus ac lobortis sem hendrerit. Etiam et sapien eget lectus interdum posuere sit amet ac urna. Aliquam pellentesque imperdiet erat, eget consectetur felis malesuada quis. Pellentesque sollicitudin, odio sed dapibus eleifend, magna sem luctus turpis, id aliquam felis dolor eu diam. Etiam ullamcorper, nunc a accumsan adipiscing, turpis odio bibendum erat, id convallis magna eros nec metus. Sed vel ligula justo, sit amet vestibulum dolor. Sed vitae augue sit amet magna ullamcorper suscipit. Quisque dictum ipsum a sapien egestas facilisis. 

\subsection{Lorem ipsum dolor sit amet, consectetur adipiscing}
Lorem ipsum dolor sit amet, consectetur adipiscing elit. Duis risus ante, auctor et pulvinar non, posuere ac lacus. Praesent egestas nisi id metus rhoncus ac lobortis sem hendrerit. Etiam et sapien eget lectus interdum posuere sit amet ac urna. Aliquam pellentesque imperdiet erat, eget consectetur felis malesuada quis. Pellentesque sollicitudin, odio sed dapibus eleifend, magna sem luctus turpis, id aliquam felis dolor eu diam.

\chapter{First Order Logic}

\begin{chapquote}{Author's name, \textit{Source of this quote}}
``This is a quote and I don't know who said this.''
\end{chapquote}

\section{Deductions}

\begin{lemma}{Universal connection to Variable Assignment Function}{forall vaf}
    \[
    \Sigma \vdash \theta \text { if and only if } \Sigma \vdash \forall x \theta
    \]
\end{lemma}

Note this lemma might seem quite strange, but note it actually makes sense, \todo{finish why}

\begin{tcolorbox}
\section*{x~\xrfill[0.3ex]{1.5pt}~Proof~\xrfill[0.3ex]{1.5pt}~x}
\begin{itemize}
    \item $ \Rightarrow $ 
    \begin{itemize}
        \item Suppose that $ \Sigma \vdash \theta$, therefore we have a deduction $ \mathcal{D}$ of $ \theta$, then the proof
            \begin{gather*}
                \mathcal{D}\\
                \left[ \left( \forall y \left( y =  y \right) \right) \land  \neg \left( \forall y \left( y =  y \right) \right) \right] \rightarrow \theta \tag{taut. PC}\\
                \left[ \left( \forall y \left( y =  y \right) \right) \land  \neg \left( \forall y \left( y =  y \right) \right) \right] \rightarrow \left(  \forall  x \right)\theta \tag{QR}\\
                \left(  \forall x \right) \theta \tag{PC}
            \end{gather*}
    \end{itemize}
    \item $ \Leftarrow $
    \begin{itemize}
        \item Suppose that $ \Sigma \vdash \forall x \theta$, so we have a deduction of it, call it  $\mathcal{D}$, then the following deduction suffices
        \begin{gather*}
           \mathcal{D} \\
           \forall x \theta\\
           \forall x \theta \rightarrow \theta _{x}^{x}\\
            \theta _{x}^{x}
        \end{gather*}
    \end{itemize}
\end{itemize}

\end{tcolorbox}


\section{Completeness}

\begin{theorem}{Completeness Theorem}{completeness}
Suppose that $\Sigma$ is a set of $\mathcal{L}$-formulas, where $ \mathcal{L}$ is a countable langauge  and $\phi$ is an $\mathcal{L}$-formula. If $\Sigma \models \phi$, then $\Sigma \vdash \phi$.

\section*{Setup}

\begin{itemize}
    \item We start by assuming that $ \Sigma \models \phi$, we must show that $ \Sigma \vdash \phi$.
    \item If $ \phi$ is not a sentence then we can always prove $ \phi'$ which is the same as $ \phi$ with all of it's variables bound
    \begin{itemize}
        \item We can do that by appending $ \left( \forall  x _{f}  \right)$ where each $ x_{f}$  is a free varaible of $ \phi$ to the front of $ \phi$ 
    \end{itemize}
\item Therefore we will prove it for all sentences $ \phi$ \todo[inline]{justify why this is equivalent}
\end{itemize}

\end{theorem}

\chapter{Topology}

\section{Topological Spaces and Continuous Functions}

\subsection{Basis for a Topology}

\begin{definition}{Basis}{basis}
If $X$ is a set, a basis for a topology on $X$ is a collection $\mathcal{B}$ of subsets of $X$ (called basis elements) such that
\begin{enumerate}
    \item For each $x \in X$, there is at least one basis element $B$ containing $x$.
    \item If $x$ belongs to the intersection of two basis elements $B_{1}$ and $B_{2}$, then there is a basis element $B_{3}$ containing $x$ such that $B_{3} \subset B_{1} \cap B_{2}$.
\end{enumerate}
If $\mathcal{B}$ satisfies these two conditions, then we define the topology $\mathcal{T}$ generated by $\mathcal{B}$ as follows: $A$ subset $U$ of $X$ is said to be open in $X$ (that is, to be an element of $\mathcal{T}$ ) if for, each $x \in U$, there is a basis element $B \in \mathcal{B}$ such that $x \in B$ and $B \subset U$. Note that each basis element is itself an element of $\mathcal{T}$.
\end{definition}


\subsection{The Subspace Topology}

\begin{definition}{Subspace Topology}{subspace_topology}
Let $X$ be a topological space with topology $\mathcal{T}$. If $Y$ is a subset of $X$, the collection
$$
\mathcal{T}_{Y}=\{Y \cap U \mid U \in \mathcal{T}\}
$$
is a topology on $Y$, called the subspace topology. With this topology, $Y$ is called a subspace of $X$; its open sets consist of all intersections of open sets of $X$ with $Y$.
\end{definition}

\subsection{The Product Topology}

\begin{definition}{Product Topology}{product_topology}
    Let $\mathcal{S}_{\beta}$ denote the collection
    $$
    \mathcal{ S } _{\beta}=\left\{\pi_{\beta}^{-1}\left(U_{\beta}\right) \mid U_{\beta} \text { open in } X_{\beta}\right\}
    $$
    and let $\mathcal{ S } $ denote the union of these collections,
    $$
    \mathcal{S}=\bigcup_{\beta \in J} \mathcal{S}_{\beta}
    $$
    The topology generated by the subbasis $\mathcal{ S } $ is called the product topology. In this topology $\prod_{\alpha \in J} X_{\alpha}$ is called a product space.
\end{definition}


\begin{theorem}{Basis for the Box Topology}{basis_for_the_box_topology}

Suppose the topology on each space $X_{\alpha}$ is given by a basis $\mathcal{B}_{\alpha}$. The collection of all sets of the form
$$
\prod_{\alpha \in \mathcal{ J } } B_{\alpha}
$$
where $B_{\alpha} \in \mathcal{B}_{\alpha}$ for each $\alpha$, will serve as a basis for the box topology on $\prod_{\alpha \in \mathcal{ J } } X_{\alpha}$.
\end{theorem}

\begin{theorem}{Basis for the Product Topology}{basis_for_the_product_topology}

Suppose the topology on each space $X_{\alpha}$ is given by a basis $\mathcal{B}_{\alpha}$. The collection of all sets of the form
\[
\prod_{\alpha \in \mathcal{ J } } B_{\alpha}
\]

where $B_{\alpha} \in \mathcal{ B } _{\alpha}$ for finitely many indices $\alpha$ and $B_{\alpha}=X_{\alpha}$ for all the remaining indices, will serve as a basis for the product topology $\prod_{\alpha \in \mathcal{ J } } X_{\alpha}$.
\end{theorem}

\begin{definition}{R Omega}{r_omega}
$\mathbb{R}^{\omega}$, the countably infinite product of $\mathbb{R}$ with itself. Recall that
$$
\mathbb{R}^{\omega}=\prod_{n \in \mathbb{N}} X_{n}
$$
with $ X_{ n }  =  \mathbb{R}  $ for each $ n $ 
\end{definition}


\subsection{The Metric Topology}

\begin{definition}{A metric}{metric}
A metric on a set $X$ is a function
$$
d: X \times X \longrightarrow \mathbb{R} 
$$
having the following properties:
\begin{enumerate}
       \item $d(x, y) \geq 0$ for all $x, y \in X$; equality holds if and only if $x=y$.
       \item $d(x, y)=d(y, x)$ for all $x, y \in X$.
       \item Triangle Inequality: $d(x, y)+d(y, z) \geq d(x, z)$, for all $x, y, z \in X$.
\end{enumerate}
\end{definition}

\begin{example}{Discrete Metric}{discrete_metric}
 $d: X \times X \rightarrow \mathbb{R}$ given by
$$
d(x, y)= \begin{cases}0 & x=y \\ 1 & \text { otherwise }\end{cases}
$$
\end{example}


\begin{definition}{Epsilon Ball}{epsilon_ball}
Given $\epsilon>0$, consider the set
$$
B_{d}(x, \epsilon)=\{y \mid d(x, y)<\epsilon\}
$$
of all points $y$ whose distance from $x$ is less than $\epsilon$. It is called the $\epsilon$-ball centered at $\boldsymbol{x}$. Sometimes we omit the metric $d$ from the notation and write this ball simply as $B(x, \epsilon)$, when no confusion will arise.
\end{definition}

\begin{definition}{Metric Topology}{metric_topology}
If $d$ is a metric on the set $X$, then the collection of all $\epsilon$-balls $B_{d}(x, \epsilon)$, for $x \in X$ and $\epsilon>0$, is a basis for a topology on $X$, called the metric topology induced by $d$.
\end{definition}


\begin{definition}{Bounded Subset of a Metric Space}{bounded}
Let $X$ be a metric space with metric $d$. A subset $A$ of $X$ is said to be bounded if there is some number $M \in  \mathbb{R}$ such that
$$
d\left(a_{1}, a_{2}\right) \leq M
$$
for every pair of points $ a_{ 1 } , a_{ 2 } \in  A $ 
\end{definition}


\section{Connectedness and Compactness}

\begin{example}{Closed and Bounded, not Compact}{}
A metric space $X$  and a closed and bounded subspace $Y$ of  $X$  that is not compact.
\end{example}

\begin{itemize}
    \item Consider the set $ X =  \left\{ \frac{1}{n}: n \in  \mathbb{N} ^{ +  }  \right\}  $, with the \hyperref[example:discrete_metric]{discrete metric}, it is bounded because the for any two points $ a, b \in X, d\left( a, b \right)  \le 1 $  \todo{closed and bounded proof}
    \item Let $ X $ be an infinite set and let consider the discrete metric on that set,  the metric topology which it induces (call it $ \mathcal{ T }  $)  is the discrete topology of $ X $. Therefore if we consider any subset $ Y $ of $ X $ it is closed, as $ X \setminus Y \in  \mathcal{ T }  $ (remember it's the discrete topology). But the open covering $ \left\{ \left\{ x \right\} : x \in  X \right\}  $ has no finite subcollection which also covers $ X $ .
\end{itemize}


\subsection{Compact Spaces}

\begin{definition}{Covering}{covering}
A collection $A$ of subsets of a space $X$ is said to cover $X$, or to be a covering of $X$, if the union of the elements of $A$ is equal to $X$. It is called an open covering of $X$ if its elements are open subsets of $X$.
\end{definition}

\begin{definition}{Compact Space}{compact_space}
A space $X$ is said to be compact if every open covering $A$ of $X$ contains a finite subcollection that also covers $X$.
\end{definition}

\begin{lemma}{Covering Yields Finite Covering if and only if Compact}{covering_yields_finite_covering_if_and_only_if_compact}
Let $Y$ be a subspace of $X$. Then $Y$ is compact if and only if every covering of $Y$ by sets open in $X$ contains a finite subcollection covering $Y$.
\end{lemma}




\end{document}
