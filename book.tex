%%%%%%%%%%%%%%%%%%%%%%%%%%%%%%%%%%%%%%%%%%%%%%%%%%%
% Document Setup
%%%%%%%%%%%%%%%%%%%%%%%%%%%%%%%%%%%%%%%%%%%%%%%%%%%

\documentclass[a4paper,11pt,oneside]{book}
\usepackage[T1]{fontenc}
\usepackage[utf8]{inputenc}
\usepackage{lmodern}
\usepackage{geometry}
% \usepackage{showframe}

%\newgeometry{vmargin={15mm}, hmargin={12mm,17mm}}   % set the margins
\geometry{left=2cm}
\geometry{right=2cm}
\setlength{\marginparwidth}{2cm}

\usepackage{knowledge}

\begin{document}

\frontmatter
\maketitle

%%%%%%%%%%%%%%%%%%%%%%%%%%%%%%%%%%%%%%%%%%%%%%%%%%%%%%%%%%%%%%%
% Add a dedication paragraph to dedicate your book to someone %
%%%%%%%%%%%%%%%%%%%%%%%%%%%%%%%%%%%%%%%%%%%%%%%%%%%%%%%%%%%%%%%
\begin{dedication}
Dedicated to the proofs that were left as exercises to the readers
\end{dedication}

%%%%%%%%%%%%%%%%%%%%%%%%%%%%%%%%%%%%%%%%%%%%%%%%%%%%%%%%%%%%%%%%%%%%%%%%
% Auto-generated table of contents, list of figures and list of tables %
%%%%%%%%%%%%%%%%%%%%%%%%%%%%%%%%%%%%%%%%%%%%%%%%%%%%%%%%%%%%%%%%%%%%%%%%
\tableofcontents
\listoffigures
\listoftables
%\listoftodos

\mainmatter

%%%%%%%%%%%
% Preface %
%%%%%%%%%%%

\chapter*{Preface}
This book contains knowledge that that me or my peers have obtained, the purpose is to explain things fundamentally and in full detail so that someone who has never touched the subject may be able to understand it. It will focus on conveying the ideas that are involved in synthesizing the new knowledge with less of a focus on the results themselves.


\section*{Structure of book}
The book is partitioned into different sections based on the domain it is involved with. There may be shared definitions and theorems throughout the chapters, but in general it will start more elementary and get more advanced.

\section*{Knowledge}
In this book you will find many results, will will characterize them as being one of the following
\begin{itemize}
    \item Theorems - Results that are of importantance and who's proof is not easily found (maybe using a novel idea)
  \item Propositions - Results of less importance who's proof could be constructed without a novel idea
  \item Lemmas - Results that are technical intermediate steps which has no standing as an independent result on first observation \footnote[2]{But sometimes they escape, as their usage becomes more than just an intermediate step,  as Zorn's or Fatou's Lemmas did}
    \item Corollaries - Results which follow readily  from an existing result of greater importance
\end{itemize}

\section*{Recommendations}
By now you might know that in order to actually get better at mathematics you have to engage with it. This book may be used as a reference at times, but I highly recommend trying to re-prove statements or coming up with your own ideas before instantly looking at the solutions.

\section*{About the companion website}
The website\footnote{\url{https://github.com/cuppajoeman/knowledge-book}} for this file contains:
\begin{itemize}
  \item A link to (freely downlodable) latest version of this document.
  \item Link to download LaTeX source for this document.
\end{itemize}

\section*{Acknowledgements}
\begin{itemize}
    \item A special word of thanks to professors who wanted to make sure I understood and learned as much as possible Alfonso Gracia-Saz\footnote{\url{https://www.math.toronto.edu/cms/alfonso-memorial/}}, Jean-Baptiste Campesato\footnote{\url{https://math.univ-angers.fr/~campesato/}}, Z-Module, riv, PlanckWalk, franciman, qergle from \#math on \url{https://libera.chat/}.
\end{itemize}

\chapter*{Contributing}
 
Contributions to the project are very welcome, let's delve into how to get started with this.

If you want to contribute to the project it's most likely that a contribution will fall into one of the following categories
\begin{itemize}
  \item Content Based 
  \begin{itemize}
      \item Adding Definitions, Theorems, \ldots
      \item Finishing TODO's
      \item Formatting of the book
  \end{itemize}
  \item Structural Layout of Project 
  \begin{itemize}
      \item Organization
      \item Simplyfing the existing structure of the directories 
      \item Making scripts which set up new structures
  \end{itemize}
  \item External
  \begin{itemize}
      \item Adding explanatory content to help onboard new users
      \item Getting others involved
      \item Creating infrastructure to support users (Github discussions)
  \end{itemize}
\end{itemize}

\section*{Content Based}

If you're looking to add content to the project \todo{finish this}

\chapter{Linear Algebra}

\section{Vectors}


\begin{definition}{Algebraic Dot Product}{algebraic_dot_product}
 Let $ u_{1} , u_{2} , \dotsc  , u_{n - 1} , u_{n}$ and $ v_{1} , v_{2} , \dotsc  , v_{n - 1} , v_{n}$ denote the components of $ \vec{u}$ and $ \vec{v}$ respectively, then we have:
\[
	\vec{v}  \cdot  \vec{u} \stackrel{\mathtt{D}}{=} \sum_{i=1}^{n} v_{i} u_{i}
\]
\end{definition}


\begin{definition}{GeometricD ot Product}{geometric_dot_product}
  Let $ \vec{u}, \vec{v} \in \mathbb{R} ^{n}$ and let $\theta$ be the angle between the two (the one in the range $ \left[ 0, \pi \right]$), then we have the dot product: 
\[
	\vec{v}  \cdot  \vec{u} \stackrel{\mathtt{D}}{=} \left\Vert v \right\Vert \left\Vert u \right\Vert \cos  \left( \theta \right) 
\]
\begin{center}
	\newcommand{\tikzAngleOfLine}{\tikz@AngleOfLine}
	  \def\tikz@AngleOfLine(#1)(#2)#3{%
	  \pgfmathanglebetweenpoints{%
	    \pgfpointanchor{#1}{center}}{%
	    \pgfpointanchor{#2}{center}}
	  \pgfmathsetmacro{#3}{\pgfmathresult}%
	  }
	\newcommand{\tikzMarkAngle}[3]{
	\tikzAngleOfLine#1#2{\AngleStart}
	\tikzAngleOfLine#1#3{\AngleEnd}
	\draw #1+(\AngleStart:0.35cm) arc (\AngleStart:\AngleEnd:0.35cm);
	}
	\usetikzlibrary{patterns,decorations.pathreplacing}
	\begin{tikzpicture}
		\coordinate (A) at (2,1);
		\coordinate (B) at (.5,2);
		\coordinate (O) at (0,0);

		\draw[->,thick,blue!60!white] (0,0) -- +(A) node [midway,below right] {$\vec v$};
		\draw[->,thick,green] (0,0) -- +(B) node [midway,above left] {$\vec u$};
		\tikzMarkAngle{(O)}{(A)}{(B)}
		\node at ($(O)+(50:.65)$) {$\theta$};
	\end{tikzpicture}
	\hspace{1cm}
	\begin{tikzpicture}
		\coordinate (A) at (-2,-1);
		\coordinate (B) at (.5,2);
		\coordinate (O) at (0,0);

		\draw[->,thick,blue!60!white] (0,0) -- +(A) node [midway,below right] {$\vec v$};
		\draw[->,thick,green] (0,0) -- +(B) node [midway,above left] {$\vec u$};
		\tikzMarkAngle{(O)}{(B)}{(A)}
		\node at ($(O)+(140:.65)$) {$\theta$};
	\end{tikzpicture}
\end{center}
\end{definition}




\begin{proposition}{Algebraic Def Implies Geometric Def of Dot Product}{algebraic_def_implies_geometric_def_of_dot_product}
    \[
	 \sum_{i=1}^{n} v_{i} u_{i} =  \left\Vert v \right\Vert \left\Vert u \right\Vert \cos  \left( \theta \right) 
    \]
\end{proposition}

\begin{proof}
    \begin{itemize}
        \item Consider two vectors $ \vec{x}, \vec{y} \in \mathbb{R} ^{2}$, we know that from the polar coordinate system we can represent them as
            \[
            \vec{x} =  \left( \left\Vert x \right\Vert \cos  \left( \theta _{x} \right), \left\Vert x \right\Vert \sin  \left( \theta _{x} \right)  \right) \quad \text{ and } \quad \vec{y} =  \left( \left\Vert y \right\Vert \cos  \left( \theta _{y} \right), \left\Vert y \right\Vert \sin  \left( \theta _{y} \right)  \right)    
            \]
        \item The the algebraic definition implies the following
            \[
            \vec{x}  \cdot  \vec{y} =  \left\Vert x \right\Vert \left\Vert y \right\Vert \cos  \left( \theta _{x} \right) \cos  \left( \theta _{y} \right)  +  \left\Vert x \right\Vert \left\Vert y \right\Vert \sin  \left( \theta  _{x} \right) \sin  \left( \theta _{y} \right) = \left\Vert x \right\Vert \left\Vert y \right\Vert \left( \cos  \left( \theta  _{x} \right) \cos  \left( \theta  _{y} \right)  +  \sin  \left( \theta _{x} \right) \sin  \left( \theta _{y} \right) \right)
            \]
        \item Then we can recall the sin angle difference formula to obtain: 
            \[
            \vec{x}  \cdot  \vec{y} = \left\Vert x \right\Vert \left\Vert y \right\Vert \left( \cos  \left( \theta_{x}  -  \theta _{y} \right) \right)
            \]
        \item And just noting that $ \theta _{x}  -  \theta _{y}$ is the angle between the vectors $ \vec{x}$ and $ \vec{y}$ as needed.
    \end{itemize}
\end{proof}

%\begin{proposition}{Scalar Factors out of Norm}
    For any vector $ \vec{v}$ and $ a \in \mathbb{R}$ we have
    \[
    \left\Vert \alpha \vec{v} \right\Vert =  \left| \alpha \right| \left\Vert  \vec{v} \right\Vert
    \]
    \begin{pf}
        \begin{align*}
            \left\Vert \alpha \vec{v} \right\Vert &=  \left\Vert \left(  \alpha \vec{v}_{1} , \alpha \vec{v}_{2} , \dotsc  , \alpha \vec{v}_{n - 1} , \alpha \vec{v}_{n} \right) \right\Vert \\
                                                  &= \sqrt{ \sum_{i=1}^{n} \left( \alpha v_{i} \right) ^{ 2}}\\
                                                  &= \sqrt{ \alpha ^{ 2} \sum_{i=1}^{n} \left(  v_{i} \right) ^{ 2}}\\
                                                  &= \left|  \alpha \right| \sqrt{ \sum_{i=1}^{n} \left(  v_{i} \right) ^{ 2}}\\
                                                  &= \left|  \alpha \right| \left\Vert \vec{v} \right\Vert
        \end{align*}
    \end{pf}
\end{proposition}

%\begin{proposition}{Scalar Times Vector is Zero implies one of them Zero}
  Suppose that $ a \in \mathbb{F}, v \in V$, and that $ av = 0$ we will prove that 
  \[
  a =  0 \in \mathbb{F} \text{ or } v = 0 \in V
  \]
  \begin{pf}
    Supposing $ a \neq 0$, then by multiplying both sides by $\frac{1}{a}$ we obtain 
    \[
    v = \left( \frac{1}{a} \right)0 = 0 \in \mathbb{F}
    \]
    by compatibility of scalar multiplication, and the fact that any scalar times the identity element is still the identity. Otherwise $ a = 0$ and the proof is done.
  \end{pf}
\end{proposition}

%\begin{proposition}{Unit Vector in the same Direction}
    Given any vector $ \vec{v}$, then the vector in the same direction as $ \vec{v}$ with length one is given by
    \[
    \vec{u} \stackrel{\mathtt{D}}{=} \frac{ \vec{v}}{ \left\Vert  \vec{v} \right\Vert}
    \]
    \begin{pf}
        \begin{itemize}
            \item By definition $ \vec{u}  =  \alpha \vec{v}$ therefore $ \vec{u} \in \operatorname{span}\left\{ \vec{v}\right\}$, since $ \left\Vert v \right\Vert \ge 0$ this forces it to point in the same direction as $  \vec{v}$ , as if $ \left\Vert  \vec{v} \right\Vert < 0$ it would be pointing in the opposite direction.
            \item  Now we will show that it is a unit vector. This follows from the fact that we can \kref{https://gitlab.com/cuppajoeman/knowledge-data/-/blob/master/Scalar_Factors_out_of_Norm-MjLKJSYDfSthY-tDwuS.pdf}{pull out scalars from norms}:
                \begin{align*}
                    \left\Vert \vec{u} \right\Vert &=  \left\Vert \frac{ \vec{v}}{ \left\Vert  \vec{v} \right\Vert} \right\Vert \\
                                                   &= \left| \frac{1}{ \left\Vert  \vec{v} \right\Vert} \right| \left\Vert  \vec{v} \right\Vert \\
                                                   &= \frac{1}{ \left\Vert  \vec{v} \right\Vert}  \left\Vert  \vec{v} \right\Vert \tag{Since $ \left\Vert \vec{v} \right\Vert \ge 0$} \\
                                                   &= 1
                \end{align*}
        \end{itemize}
    \end{pf}
\end{proposition}

%\begin{proposition}{Vector Component Formula}
    Let $ \vec{u}$ and $ \vec{v} \neq \vec{0}$ be vectors, then
    \[
    \operatorname{v-comp}_{ \vec{v}} \vec{u} =  \left( \frac{ \vec{v}  \cdot  \vec{u}}{ \vec{v }  \cdot \vec{v}} \right) \vec{v}
    \]
    \begin{pf}
        \begin{itemize}
            \item From the definition of $ \operatorname{v-comp}_{ \vec{v}} \vec{u}$, we know that it's in the direction of $ \vec{v}$, that is, for some $ \alpha \in \mathbb{R}$ 
            \[
            \operatorname{v-comp}_{ \vec{v}} \vec{u} =  \alpha \vec{v}
            \]
            \item Additionally we know that $ \vec{u }  -  \operatorname{v-comp}_{ \vec{v}} \vec{u}$ is orthogonal to $ \vec{v}$, so we have:
                \[
                    \left( \vec{u }  -  \operatorname{v-comp}_{ \vec{v}} \vec{u} \right)  \cdot \vec{v} =  0
                \]
            \item We will sub in  $  \operatorname{v-comp}_{ \vec{v}} \vec{u} =  \alpha \vec{v}$ and then isolate for $ \alpha$.
                \begin{align*}
                    \left( \vec{u }  -  \operatorname{v-comp}_{ \vec{v}} \vec{u} \right)  \cdot \vec{v} =  0 &\Leftrightarrow  \left( \vec{ u}  -  \alpha \vec{v} \right)  \cdot v =  0 \\
                                                                                                             &\Leftrightarrow \vec{u}  \cdot \vec{v}  -  \alpha \vec{v}  \cdot \vec{v}   = 0\\
                                                                                                             &\Leftrightarrow \boxed{\alpha =  \frac{ \vec{u}  \cdot  \vec{v}}{ \vec{v}  \cdot  \vec{v}}}
                \end{align*}
            \item Thus we know 
                \[
                    \boxed{ \operatorname{v-comp}_{ \vec{v}} \vec{u} =  \left( \frac{ \vec{u}  \cdot  \vec{v}}{ \vec{v}  \cdot  \vec{v}} \right) \vec{v}}
                \]
        \end{itemize}
    \end{pf}
\end{proposition}



\section{Matrices}

\begin{definition}{Matrix Multiplication}{matrix_multiplication}
    If $\mathbf{A}$ is an $m \times n$ matrix and $\mathbf{B}$ is an $n \times p$ matrix,
    \[
    \mathbf{A}=\left(\begin{array}{cccc}
        a_{11} & a_{12} & \cdots & a_{1 n} \\
        a_{21} & a_{22} & \cdots & a_{2 n} \\
        \vdots & \vdots & \ddots & \vdots \\
        a_{m 1} & a_{m 2} & \cdots & a_{m n}
        \end{array}\right), \quad \mathbf{B}=\left(\begin{array}{cccc}
        b_{11} & b_{12} & \cdots & b_{1 p} \\
        b_{21} & b_{22} & \cdots & b_{2 p} \\
        \vdots & \vdots & \ddots & \vdots \\
        b_{n 1} & b_{n 2} & \cdots & b_{n p}
    \end{array}\right)
    \]
    the matrix product $\mathbf{C}=\mathbf{A B}$ (denoted without multiplication signs or dots) is defined to be the $m \times p$ matrix 
    \[
    \mathbf{C}=\left(\begin{array}{cccc}
    c_{11} & c_{12} & \cdots & c_{1 p} \\
    c_{21} & c_{22} & \cdots & c_{2 p} \\
    \vdots & \vdots & \ddots & \vdots \\
    c_{m 1} & c_{m 2} & \cdots & c_{m p}
    \end{array}\right)
    \]
    such that
    \[
    c_{i j}=a_{i 1} b_{1 j}+a_{i 2} b_{2 j}+\cdots+a_{i n} b_{n j}=\sum_{k=1}^{n} a_{i k} b_{k j}
    \]
    for $i=1, \ldots, m$ and $j=1, \ldots, p$.
\end{definition}


\lstinputlisting[language=Python]{linear_algebra/programs/matrix_multiplication.py}


\chapter{First Order Logic}

\begin{chapquote}{Author's name, \textit{Source of this quote}}
``This is a quote and I don't know who said this.''
\end{chapquote}

\section{Deductions}

\begin{lemma}{Universal connection to Variable Assignment Function}{forall vaf}
    \[
    \Sigma \vdash \theta \text { if and only if } \Sigma \vdash \forall x \theta
    \]
\end{lemma}

Note this lemma might seem quite strange, but note it actually makes sense, \todo{finish why}

\begin{tcolorbox}
\section*{x~\xrfill[0.3ex]{1.5pt}~Proof~\xrfill[0.3ex]{1.5pt}~x}
\begin{itemize}
    \item $ \Rightarrow $ 
    \begin{itemize}
        \item Suppose that $ \Sigma \vdash \theta$, therefore we have a deduction $ \mathcal{D}$ of $ \theta$, then the proof
            \begin{gather*}
                \mathcal{D}\\
                \left[ \left( \forall y \left( y =  y \right) \right) \land  \neg \left( \forall y \left( y =  y \right) \right) \right] \rightarrow \theta \tag{taut. PC}\\
                \left[ \left( \forall y \left( y =  y \right) \right) \land  \neg \left( \forall y \left( y =  y \right) \right) \right] \rightarrow \left(  \forall  x \right)\theta \tag{QR}\\
                \left(  \forall x \right) \theta \tag{PC}
            \end{gather*}
    \end{itemize}
    \item $ \Leftarrow $
    \begin{itemize}
        \item Suppose that $ \Sigma \vdash \forall x \theta$, so we have a deduction of it, call it  $\mathcal{D}$, then the following deduction suffices
        \begin{gather*}
           \mathcal{D} \\
           \forall x \theta\\
           \forall x \theta \rightarrow \theta _{x}^{x}\\
            \theta _{x}^{x}
        \end{gather*}
    \end{itemize}
\end{itemize}

\end{tcolorbox}


\section{Completeness}

\begin{theorem}{Completeness Theorem}{completeness}
Suppose that $\Sigma$ is a set of $\mathcal{L}$-formulas, where $ \mathcal{L}$ is a countable langauge  and $\phi$ is an $\mathcal{L}$-formula. If $\Sigma \models \phi$, then $\Sigma \vdash \phi$.

\section*{Setup}

\begin{itemize}
    \item We start by assuming that $ \Sigma \models \phi$, we must show that $ \Sigma \vdash \phi$.
    \item If $ \phi$ is not a sentence then we can always prove $ \phi'$ which is the same as $ \phi$ with all of it's variables bound
    \begin{itemize}
        \item We can do that by appending $ \left( \forall  x _{f}  \right)$ where each $ x_{f}$  is a free varaible of $ \phi$ to the front of $ \phi$ 
    \end{itemize}
\item Therefore we will prove it for all sentences $ \phi$ \todo[inline]{justify why this is equivalent}
\end{itemize}

\end{theorem}


\chapter{Topology}

\section{Topological Spaces and Continuous Functions}

\subsection{Basis for a Topology}

\begin{definition}{Basis}{basis}
    If $X$ is a set, a basis for a topology on $X$ is a collection $\mathcal{B}$ of subsets of $X$ (called basis elements) such that
    \begin{enumerate}
        \item For each $x \in X$, there is at least one basis element $B$ containing $x$.
        \item If $x$ belongs to the intersection of two basis elements $B_{1}$ and $B_{2}$, then there is a basis element $B_{3}$ containing $x$ such that $B_{3} \subset B_{1} \cap B_{2}$.
    \end{enumerate}
    If $\mathcal{B}$ satisfies these two conditions, then we define the topology $\mathcal{T}$ generated by $\mathcal{B}$ as follows: $A$ subset $U$ of $X$ is said to be open in $X$ (that is, to be an element of $\mathcal{T}$) if for, each $x \in U$, there is a basis element $B \in \mathcal{B}$ such that $x \in B$ and $B \subset U$. Note that each basis element is itself an element of $\mathcal{T}$.
\end{definition}


\subsection{The Subspace Topology}

\begin{definition}{Subspace Topology}{subspace_topology}
    Let $X$ be a topological space with topology $\mathcal{T}$. If $Y$ is a subset of $X$, the collection
    \[
    \mathcal{T}_{Y}=\{Y \cap U \mid U \in \mathcal{T}\}
    \]
    is a topology on $Y$, called the subspace topology. With this topology, $Y$ is called a subspace of $X$; its open sets consist of all intersections of open sets of $X$ with $Y$.
\end{definition}


\subsection{The Product Topology}

\begin{definition}{Product Topology}{product_topology}
    Let $\mathcal{S}_{\beta}$ denote the collection
    $$
    \mathcal{ S } _{\beta}=\left\{\pi_{\beta}^{-1}\left(U_{\beta}\right) \mid U_{\beta} \text { open in } X_{\beta}\right\}
    $$
    and let $\mathcal{ S } $ denote the union of these collections,
    $$
    \mathcal{S}=\bigcup_{\beta \in J} \mathcal{S}_{\beta}
    $$
    The topology generated by the subbasis $\mathcal{ S } $ is called the product topology. In this topology $\prod_{\alpha \in J} X_{\alpha}$ is called a product space.
\end{definition}


\begin{theorem}{Basis for the Box Topology}{basis_for_the_box_topology}

Suppose the topology on each space $X_{\alpha}$ is given by a basis $\mathcal{B}_{\alpha}$. The collection of all sets of the form
$$
\prod_{\alpha \in \mathcal{ J } } B_{\alpha}
$$
where $B_{\alpha} \in \mathcal{B}_{\alpha}$ for each $\alpha$, will serve as a basis for the box topology on $\prod_{\alpha \in \mathcal{ J } } X_{\alpha}$.
\end{theorem}

\begin{theorem}{Basis for the Product Topology}{basis_for_the_product_topology}

Suppose the topology on each space $X_{\alpha}$ is given by a basis $\mathcal{B}_{\alpha}$. The collection of all sets of the form
\[
\prod_{\alpha \in \mathcal{ J } } B_{\alpha}
\]

where $B_{\alpha} \in \mathcal{ B } _{\alpha}$ for finitely many indices $\alpha$ and $B_{\alpha}=X_{\alpha}$ for all the remaining indices, will serve as a basis for the product topology $\prod_{\alpha \in \mathcal{ J } } X_{\alpha}$.
\end{theorem}

\begin{definition}{R Omega}{r_omega}
$\mathbb{R}^{\omega}$, the countably infinite product of $\mathbb{R}$ with itself. Recall that
$$
\mathbb{R}^{\omega}=\prod_{n \in \mathbb{N}} X_{n}
$$
with $ X_{ n }  =  \mathbb{R}  $ for each $ n $ 
\end{definition}


\subsection{The Metric Topology}

\begin{definition}{A metric}{metric}
A metric on a set $X$ is a function
$$
d: X \times X \longrightarrow \mathbb{R} 
$$
having the following properties:
\begin{enumerate}
       \item $d(x, y) \geq 0$ for all $x, y \in X$; equality holds if and only if $x=y$.
       \item $d(x, y)=d(y, x)$ for all $x, y \in X$.
       \item Triangle Inequality: $d(x, y)+d(y, z) \geq d(x, z)$, for all $x, y, z \in X$.
\end{enumerate}
\end{definition}

\begin{example}{Discrete Metric}{discrete_metric}
 $d: X \times X \rightarrow \mathbb{R}$ given by
$$
d(x, y)= \begin{cases}0 & x=y \\ 1 & \text { otherwise }\end{cases}
$$
\end{example}


\begin{definition}{Epsilon Ball}{epsilon_ball}
Given $\epsilon>0$, consider the set
$$
B_{d}(x, \epsilon)=\{y \mid d(x, y)<\epsilon\}
$$
of all points $y$ whose distance from $x$ is less than $\epsilon$. It is called the $\epsilon$-ball centered at $\boldsymbol{x}$. Sometimes we omit the metric $d$ from the notation and write this ball simply as $B(x, \epsilon)$, when no confusion will arise.
\end{definition}

\begin{definition}{Metric Topology}{metric_topology}
If $d$ is a metric on the set $X$, then the collection of all $\epsilon$-balls $B_{d}(x, \epsilon)$, for $x \in X$ and $\epsilon>0$, is a basis for a topology on $X$, called the metric topology induced by $d$.
\end{definition}


\begin{definition}{Bounded Subset of a Metric Space}{bounded}
Let $X$ be a metric space with metric $d$. A subset $A$ of $X$ is said to be bounded if there is some number $M \in  \mathbb{R}$ such that
$$
d\left(a_{1}, a_{2}\right) \leq M
$$
for every pair of points $ a_{ 1 } , a_{ 2 } \in  A $ 
\end{definition}


\section{Connectedness and Compactness}

\begin{example}{Closed and Bounded, not Compact}{}
A metric space $X$  and a closed and bounded subspace $Y$ of  $X$  that is not compact.
\end{example}

\begin{itemize}
    \item Consider the set $ X =  \left\{ \frac{1}{n}: n \in  \mathbb{N} ^{ +  }  \right\}  $, with the \hyperref[example:discrete_metric]{discrete metric}, it is bounded because the for any two points $ a, b \in X, d\left( a, b \right)  \le 1 $  \todo{closed and bounded proof}
    \item Let $ X $ be an infinite set and let consider the discrete metric on that set,  the metric topology which it induces (call it $ \mathcal{ T }  $)  is the discrete topology of $ X $. Therefore if we consider any subset $ Y $ of $ X $ it is closed, as $ X \setminus Y \in  \mathcal{ T }  $ (remember it's the discrete topology). But the open covering $ \left\{ \left\{ x \right\} : x \in  X \right\}  $ has no finite subcollection which also covers $ X $ .
\end{itemize}


\subsection{Compact Spaces}

\begin{definition}{Covering}{covering}
A collection $A$ of subsets of a space $X$ is said to cover $X$, or to be a covering of $X$, if the union of the elements of $A$ is equal to $X$. It is called an open covering of $X$ if its elements are open subsets of $X$.
\end{definition}

\begin{definition}{Compact Space}{compact_space}
A space $X$ is said to be compact if every open covering $A$ of $X$ contains a finite subcollection that also covers $X$.
\end{definition}

\begin{lemma}{Covering Yields Finite Covering if and only if Compact}{covering_yields_finite_covering_if_and_only_if_compact}
Let $Y$ be a subspace of $X$. Then $Y$ is compact if and only if every covering of $Y$ by sets open in $X$ contains a finite subcollection covering $Y$.
\end{lemma}


\end{document}
