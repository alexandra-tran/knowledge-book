
\chapter{Topology}

\section{Topological Spaces and Continuous Functions}

\subsection{Basis for a Topology}

\begin{definition}{Basis}{basis}
    If $X$ is a set, a basis for a topology on $X$ is a collection $\mathcal{B}$ of subsets of $X$ (called basis elements) such that
    \begin{enumerate}
        \item For each $x \in X$, there is at least one basis element $B$ containing $x$.
        \item If $x$ belongs to the intersection of two basis elements $B_{1}$ and $B_{2}$, then there is a basis element $B_{3}$ containing $x$ such that $B_{3} \subset B_{1} \cap B_{2}$.
    \end{enumerate}
    If $\mathcal{B}$ satisfies these two conditions, then we define the topology $\mathcal{T}$ generated by $\mathcal{B}$ as follows: $A$ subset $U$ of $X$ is said to be open in $X$ (that is, to be an element of $\mathcal{T}$) if for, each $x \in U$, there is a basis element $B \in \mathcal{B}$ such that $x \in B$ and $B \subset U$. Note that each basis element is itself an element of $\mathcal{T}$.
\end{definition}


\subsection{The Subspace Topology}

\begin{definition}{Subspace Topology}{subspace_topology}
    Let $X$ be a topological space with topology $\mathcal{T}$. If $Y$ is a subset of $X$, the collection
    \[
    \mathcal{T}_{Y}=\{Y \cap U \mid U \in \mathcal{T}\}
    \]
    is a topology on $Y$, called the subspace topology. With this topology, $Y$ is called a subspace of $X$; its open sets consist of all intersections of open sets of $X$ with $Y$.
\end{definition}


\subsection{The Product Topology}


\begin{definition}{Product Topology}{product_topology}
    Let $\mathcal{S}_{\beta}$ denote the collection
    \[
    \mathcal{ S } _{\beta}=\left\{\pi_{\beta}^{-1}\left(U_{\beta}\right) \mid U_{\beta} \text { open in } X_{\beta}\right\}
    \]
    and let $\mathcal{ S } $ denote the union of these collections,
    \[
    \mathcal{S}=\bigcup_{\beta \in J} \mathcal{S}_{\beta}
    \]
    The topology generated by the subbasis $\mathcal{ S } $ is called the product topology. In this topology $\prod_{\alpha \in J} X_{\alpha}$ is called a product space.
\end{definition}


\begin{definition}{Box Topology}{box_topology}
Let $\left\{X_{\alpha}\right\}_{\alpha \in J}$ be an indexed family of topological spaces. Let us take as a basis for a topology on the product space
$$
\prod_{\alpha \in J} X_{\alpha}
$$
the collection of all sets of the form
$$
\prod_{\alpha \in J} U_{\alpha}
$$
where $U_{\alpha}$ is open in $X_{\alpha}$, for each $\alpha \in J$. The topology generated by this basis is called the box topology.
\end{definition}


\begin{theorem}{Basis for the Box Topology}{basis_for_the_box_topology}

Suppose the topology on each space $X_{\alpha}$ is given by a basis $\mathcal{B}_{\alpha}$. The collection of all sets of the form
\[
\prod_{\alpha \in \mathcal{ J } } B_{\alpha}
\]
where $B_{\alpha} \in \mathcal{B}_{\alpha}$ for each $\alpha$, will serve as a basis for the box topology on $\prod_{\alpha \in \mathcal{ J } } X_{\alpha}$.
\end{theorem}

\begin{theorem}{Basis for the Product Topology}{basis_for_the_product_topology}

Suppose the topology on each space $X_{\alpha}$ is given by a basis $\mathcal{B}_{\alpha}$. The collection of all sets of the form
\[
\prod_{\alpha \in \mathcal{ J } } B_{\alpha}
\]

where $B_{\alpha} \in \mathcal{ B } _{\alpha}$ for finitely many indices $\alpha$ and $B_{\alpha}=X_{\alpha}$ for all the remaining indices, will serve as a basis for the product topology $\prod_{\alpha \in \mathcal{ J } } X_{\alpha}$.
\end{theorem}

\begin{definition}{R Omega}{r_omega}
$\mathbb{R}^{\omega}$, the countably infinite product of $\mathbb{R}$ with itself. Recall that
\[
    \mathbb{R}^{\omega}=\prod_{n \in \mathbb{N}} X_{n}
\]
with $ X_{ n }  =  \mathbb{R}  $ for each $ n $ 
\end{definition}


\subsection{The Metric Topology}

\begin{definition}{A metric}{metric}
A metric on a set $X$ is a function
\[
    d: X \times X \to \mathbb{R} 
\]
having the following properties:
\begin{enumerate}
       \item $d(x, y) \geq 0$ for all $x, y \in X$; equality holds if and only if $x=y$.
       \item $d(x, y)=d(y, x)$ for all $x, y \in X$.
       \item Triangle Inequality: $d(x, y)+d(y, z) \geq d(x, z)$, for all $x, y, z \in X$.
\end{enumerate}
\end{definition}

\begin{example}{Discrete Metric}{discrete_metric}
 $d: X \times X \rightarrow \mathbb{R}$ given by

\[
d(x, y)= \begin{cases}0 & x=y \\ 1 & \text { otherwise }\end{cases}
\]
\end{example}


\begin{definition}{Epsilon Ball}{epsilon_ball}
Given $\epsilon>0$, consider the set
\[
B_{d}(x, \epsilon)=\{y \mid d(x, y)<\epsilon\}
\]
of all points $y$ whose distance from $x$ is less than $\epsilon$. It is called the $\epsilon$-ball centered at $\boldsymbol{x}$. Sometimes we omit the metric $d$ from the notation and write this ball simply as $B(x, \epsilon)$, when no confusion will arise.
\end{definition}


\begin{lemma}{Epsilon Ball Contains Another}{epsilon_ball_contains_another}
    Let $ x \in X $ then for every $ B\left( x, \varepsilon  \right)  $ there is $ y \in B\left( x, \varepsilon  \right)  $ and $ \delta \in  \mathbb{R} ^{ +  }  $ such that 
    \[
    B\left( y, \delta  \right) \subseteq B\left( x, \varepsilon  \right) 
    \]
\end{lemma}
\begin{proof}
    \begin{itemize}
        \item Let $ y \in  B\left( x, \varepsilon \right)  $ we claim  $ B\left( y, \delta  \right)  $ where $ \delta =  \varepsilon -  d\left( x,y \right)  $ works.
        \item Note $ B\left( y, \delta  \right) \subseteq B\left( x, \varepsilon  \right)  $, since given any $ z \in B\left( y, \delta  \right)  $ we have that $ d\left( y, z \right) < \varepsilon  - d\left( x,y \right)$, re-arranging gives us $ d\left( x,y \right) +  d\left( y, z \right) < \varepsilon   $ and by the triangle inequality we get $ d\left( x, z  \right) < \varepsilon  $ therefore $ z \in  B\left( x, \varepsilon  \right)  $
    \end{itemize}
\end{proof}


\begin{proposition}{Epsilon Balls Form A Basis}{epsilon_balls_form_a_basis}
The collection $ \mathcal{ B }   $ of all $\epsilon$-balls $B_{d}(x, \epsilon)$, for $x \in X$ and $\epsilon>0$, is a basis for a topology on $X$
\end{proposition}
\begin{proof}
    \begin{itemize}
        \item Let $ x \in  X $ then clearly $ x \in  B _{ d }  \left( x, \varepsilon    \right)  $ works for any $ \varepsilon  \in  \mathbb{R} ^{ +  }   $ 
        \item Let $  B _{ 1 } , B _{ 2 } \in  \mathcal{ B }    $ and and let $ y \in  B _{ 1 } \cap B _{ 2 }  $, for each of $ B _{ 1 }  $ and $ B _{ 2 }  $ \hyperref[lemma:epsilon_ball_contains_another]{we have} $ \delta _{ 1 }  $ and $ \delta _{ 2 }  $ respectively with $ B\left( y, \delta _{ 1 }  \right) \subseteq B _{ 1 }  $ and $ B\left( y, \delta _{ 2 }  \right) \subseteq B _{ 2 }  $, by taking $ \delta =  \min\left( \delta _{ 1 }  , \delta _{ 2 }   \right)  $ then we have $ B\left( y, \delta  \right) \subseteq B _{ 1 } \cap  B _{ 2 }   $ as needed.
    \end{itemize}
\end{proof}



\begin{definition}{Metric Topology}{metric_topology}
If $d$ is a metric on the set $X$, then the collection of all $\epsilon$-balls $B_{d}(x, \epsilon)$, for $x \in X$ and $\epsilon>0$, is a basis for a topology on $X$, called the metric topology induced by $d$.
\end{definition}


\begin{definition}{Bounded Subset of a Metric Space}{bounded}
Let $X$ be a metric space with metric $d$. A subset $A$ of $X$ is said to be bounded if there is some number $M \in  \mathbb{R}$ such that
\[
d\left(a_{1}, a_{2}\right) \leq M
\]
for every pair of points $ a_{ 1 } , a_{ 2 } \in  A $ 
\end{definition}


\begin{example}{Function on R omega}{function_on_r_omega}
    Consider a function $ h : \mathbb{R} ^{ \omega  }  \to \mathbb{R} ^{ \omega  }  $ defined by 
    \[
    h\left( x_{1} , x_{2} , \ldots \right) = \left( \alpha _{ 1 } x _{ 1 } , \alpha _{ 2 } x _{ 2 }, \ldots  \right)
    \]
    with $ \alpha _{ 1 } , \alpha _{ 2 } , \ldots  \in  \mathbb{R} \setminus \left\{ 0 \right\}  $ 
    \begin{enumerate}
        \item Is $ h $  continuous, when $ \mathbb{R} ^{ \omega  }  $  is given the product topology?

        \item Is $ h $  continuous, when $ \mathbb{R} ^{ \omega  }  $ is given the box topology?

        \item Is $ h $  continuous, when $ \mathbb{R} ^{ \omega  }  $ is given the uniform topology?
    \end{enumerate}
\end{example}
\begin{proof}
    \begin{enumerate}
        \item 
        \begin{itemize}
            \item Take an arbitrary basis element from the product topology, that is: 
            \[
            \prod_{\alpha \in \mathcal{ J } } B_{\alpha}
            \]
            where $B_{\alpha} \in \mathcal{ B } _{\mathbb{R} }$ for finitely many indices $\alpha$ and $B_{\alpha}=\mathbb{R} $ for all the remaining indices.
        \item[$ \alpha:$]  Now note the following
            \begin{itemize}
                \item The inverse image of each of these intervals is a scaled version of itself therefore it is still an interval of $ \mathbb{R}  $, and it is therefore still open with respect to $ \mathbb{R}$ 
                \item The inverse image of $ \mathbb{R}  $ under any scaling is still $ \mathbb{R}  $ 
            \end{itemize}
            \item Therefore $ h ^{-1} \left( \prod _{ \alpha  \in \mathcal{ J }   } B _{ \alpha  }  \right)  $ is a product of finitely many intervals, and infinitely many $ \mathbb{R}  $'s and so it is open with respect to $ \mathbb{R} ^{ \omega  }  $ as needed.
        \end{itemize}
        \item 
        \begin{itemize}
            \item Given an arbitrary basis element of the box topology we have
            \[
            \prod_{\alpha \in \mathcal{ J } } B_{\alpha}
            \]
            where $B_{\alpha} \in \mathcal{B}_{ \mathbb{R} }$ for each $\alpha$
            \item Due to $ \alpha  $ we know that each of the intervals become new scaled intervals and so $ h ^{-1} \left( \prod _{ \alpha \in \mathcal{ J } B _{ \alpha  }   }  \right)  $ is a product of open intervals and is therefore open in with respect $ \mathbb{R} ^{ \omega  }  $ equipped with the box topology 
        \end{itemize}
        \item Recall that a basis for the uniform topology are epsilon balls with radius less then $ 1 $ if that's the case, then so long as every $ \alpha _{ i }  $ is greater than $ 1 $ then the inverse image scales them to be even smaller, resulting in an open set, otherwise the function $ h $ wouldn't be continuous, since all the $ \alpha _{ i }  $'s are greater than one, it is continouous.
    \end{enumerate}
\end{proof}




\section{Connectedness and Compactness}


\begin{definition}{Connected Space}{connected}
Let $X$ be a topological space. A separation of $X$ is a pair $U, V$ of disjoint nonempty open subsets of $X$ whose union is $X$. The space $X$ is said to be connected if there does not exist a separation of $X$.
\end{definition}


Notice that $ U, V $ are actually clopen, as $ X \setminus  U =  V $ and $ X \setminus  V = U $ stating that $ V $ and $ U $ are closed as well.

\begin{example}{Closed and Bounded, not Compact}{closed_and_bounded_not_compact}
A metric space $X$  and a closed and bounded subspace $Y$ of  $X$  that is not compact.
\end{example}


\begin{itemize}
    \item Consider the set $ X =  \left\{ \frac{1}{n}: n \in  \mathbb{N} ^{ +  }  \right\}  $, with the \hyperref[example:discrete_metric]{discrete metric}, it is bounded because the for any two points $ a, b \in X, d\left( a, b \right)  \le 1 $  %todo{closed and bounded proof}
    \item Let $ X $ be an infinite set and let consider the discrete metric on that set,  the metric topology which it induces (call it $ \mathcal{ T }  $)  is the discrete topology of $ X $. Therefore if we consider any subset $ Y $ of $ X $ it is closed, as $ X \setminus Y \in  \mathcal{ T }  $ (remember it's the discrete topology). But the open covering $ \left\{ \left\{ x \right\} : x \in  X \right\}  $ has no finite subcollection which also covers $ X $.
\end{itemize}

\begin{example}{R omega Connected?}{r_omega_connected?}
Consider the product, uniform, and box topologies on $\mathbb{R} ^{\omega}$. In which topologies is $\mathbb{R} ^{\omega}$ connected?
\end{example}
\begin{proof}
    \begin{enumerate}
        \item 
        \begin{itemize}
            \item Consider the product topology, we will show that $ \mathbb{R} ^{ \omega  }  $ is not connected. Consider the set $ A $ of \hyperref[definition:bounded_sequence]{bounded sequences} in $ \mathbb{R} ^{ \omega  }  $ , and the complement of $ A $, namely the unbounded sequence of $ \mathbb{R} ^{ \omega  }  $ let's label this set as $ B =  \mathbb{R}  ^{ \omega  } \setminus  A  $.
            \begin{itemize}
                \item Note that $ A $ is open in the box topology as if we fix any $ \varepsilon > 0 $ we may define for any $ \vec{x}  \in  \mathbb{R} ^{ \omega  }  $ 
                    \[
                    U _{ \vec{x}  }  \stackrel{\mathtt{D}}{=} \left( x _{ 1 } -  \varepsilon , x _{ 1 } +  \varepsilon  \right) \times \left( x _{ 2 } -  \varepsilon , x _{ 2 } +  \varepsilon  \right) \times \ldots
                    \]
                \item If $ \vec{a}  $ is a bounded sequence then $ U $ is a set of bounded sequences, as it only ever adds constants to the terms of $ \vec{a}  $, therefore $ \vec{a}  \in  U _{ \vec{a}  } $ and $ U _{ \vec{a}  }  \subseteq A $, therefore $ A $ is open.
                \item If $ \vec{b}  $ is an unbounded sequence then $ U $ is a set of unbounded sequences, as adding a constant to every element to an unbounded sequence yields an unbounded sequence. So for every $ \vec{b} \in  B  $ there we have $ \vec{b} \in  U _{ \vec{b}  }  $ and $ U _{ \vec{b}  } \subseteq B $ so $ B $ is open.
            \end{itemize}
            \item Therefore $ A $ non-trivial set which is both open and closed in $ \mathbb{R} ^{ \omega  }  $ and so it is disconnected.
        \end{itemize}
        \item Note that $ \mathbb{R} ^{ \omega  }  $ is closed with the uniform topology by following the same proof with $ \varepsilon = 1 $ 
        \item 
        \begin{itemize}
            \item Now if we consider $ \mathbb{R} ^{ \omega  }  $ with the product topology we will see that $ \mathbb{R} ^{ \omega  }  $ is connected.
            \item Let $ \mathbb{R} _{ 0 }^{ n } \stackrel{\mathtt{D}}{=} \left\{ \left( x _{ 1 } , x _{ 2 } , \ldots \right): ~\text{where for}~  i > n  ~\text{we have that}~   x _{ i } = 0   \right\}    $ 
            \item $ \mathbb{R} _{ 0 }^{ n }   $ is homeomorphic to $ \mathbb{R} ^{ n }  $ and since $ \mathbb{R}  $ is connected, and the finite cartesian product of connected spaces is connected, we get that $ \mathbb{R} _{ 0 }^{ n }  $ is connected.
            \item We now will show that the closure of $ \mathbb{R} ^{ \infty  }  $ is equal to $ \mathbb{R} ^{ \omega  }  $ and then by the fact that the closure of a connected set is closed we obtain that $ \mathbb{R} ^{  \omega  }  $ is connected.
            \item To show that $ R ^{ \omega  }  $ is equal to the closure of $ \mathbb{R} ^{ \infty  }  $ we proceed by using the fact that a point is part of the closure of a set if and only if every basis element intersects it:
            \begin{itemize}
                \item Let $ \vec{a} \in \mathbb{R} ^{ \omega  }   $ and let $ B =  \prod _{ i \in \mathbb{N}   }  B _{ i }  $ be some basis element for the product topology with $ \vec{a}  \in  U $, now we have to show this set intersects $ \mathbb{R} ^{ \infty  }  $. Since only finitely many of the $ B _{ i }  $ are equal to basis elements and the rest are equal to $ \mathbb{R}  $ that means there is some $ N \in  \mathbb{N}  $ such that $ \forall j \in  \mathbb{N} ^{ \ge N }, B _{ j } = \mathbb{R}   $, therefore the point $ \vec{b} = \left( a_{1} , a_{2} , \dotsc  , a_{N - 1} , a_{N}, 0, 0, 0, \ldots \right) \in B \cap \mathbb{R} ^{ \infty  }  $
            \end{itemize}
        \end{itemize}
    \end{enumerate}
\end{proof}



\begin{proposition}{Connected Implies Closure Connected}{connected_implies_closure_connected}
    Let $ A \subseteq X $ be a connected subspace of $ X $ then $ \overline{A}  $ is also connected.
\end{proposition}
\begin{proof}
    \begin{itemize}
        \item Suppose for the sake of contradiction that there is a separation $ B, C $  of $ \overline{A}  $, then $  B \cup  C =  \overline{A}  $ and note that that means that $ A \subseteq B \cup  C $ since $ A \subseteq \overline{A}  $  so $ \left( B \cup C \right) \cap  A =  \overline{A}  \cap  A =  A $. 
        \item Therefore $ \left( B \cap A \right) \cup  \left( C \cap  A \right) =  A$ is a separation of $ A $ noting that $ B \cap A $ and $ C \cap A $ are non-empty because ...
    \end{itemize}
\end{proof}


\begin{definition}{Totally Disconnected}{totally_disconnected}
    A topological space is totally disconnected if it's only connected subspaces are one-point sets.
\end{definition}


Consider $ \mathbb{R} _{ \ell } $ if we have $ \left\{ a \right\}  $ and $ \left\{ b \right\}  $ then the open sets $ \left( - \infty , b \right) $ and $ \left[ b, \infty  \right) $ is a separation of $ \mathbb{R} _{ \ell }  $, therefore it's disconnected. Similarly for any two points $ a, b $  in $ \mathbb{Q}  $ we have some irrational number $ r $ between the two, and thus $ \left( - \infty , r \right) _{ \mathbb{Q}  } , \left( r, \infty  \right) _{ \mathbb{Q}  }  $ is a separation thus $ \mathbb{Q}  $ is totally disconnected under this topology. 

Let's look at $ \mathbb{R}  $ with the fite complement topology. Right off the bat, we note that if a set is finite in $ \mathbb{R}  $ it's complement must be infinite therefore if $ \mathbb{R}  $ was completely disconnected it would mean for any singleton sets, we have a separation $ U, V $, but that means that $ U $ and $ V $ must be infinite, but we then get a contraditiction as $ \mathbb{R} =  U \cup V $ so $ R \setminus U = V $, now since $ U $ was open this implies that $ V $ is finite, which is a contradiction. This idea may be extended to $ \mathbb{R} ^{ 2 }  $.


\subsection{Compact Spaces}

\begin{definition}{Covering}{covering}
A collection $A$ of subsets of a space $X$ is said to cover $X$, or to be a covering of $X$, if the union of the elements of $A$ is equal to $X$. It is called an open covering of $X$ if its elements are open subsets of $X$.
\end{definition}

\begin{definition}{Compact Space}{compact_space}
A space $X$ is said to be compact if every open covering $A$ of $X$ contains a finite subcollection that also covers $X$.
\end{definition}

\begin{lemma}{Covering Yields Finite Covering if and only if Compact}{covering_yields_finite_covering_if_and_only_if_compact}
Let $Y$ be a subspace of $X$. Then $Y$ is compact if and only if every covering of $Y$ by sets open in $X$ contains a finite subcollection covering $Y$.
\end{lemma}
