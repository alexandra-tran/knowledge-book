\chapter{Analysis}

\begin{theorem}{Triangle Inequality}{triangle_inequality}
    Let $ a, b \in \mathbb{R}$ then 
    \[
    \left| a  +  b \right| \le \left| a \right|  +  \left| b \right|
    \]
\end{theorem}
\begin{proof}
    \begin{itemize}
        \item $ a  +  b \le \left| a \right|  +  b \le \left| a  \right|  +  \left| b \right| \Leftrightarrow a  +  b \le \left| a \right|  +  \left| b \right|$ 
        \item $ - \left( a  +  b  \right) =   - a    - b \le \left|  a \right|   - b \le \left|   a \right|  +  \left|   b \right| \Leftrightarrow -\left( a  + b \right) \le \left| a \right|  +  \left| b \right|$ 
        \begin{itemize}
            \item Which are both derived by the fact that \hyperref[lemma:absolute_value_is_equal_to_max]{$ \left|x  \right| =  \max\left(x ,  - x\right)$} $\ge x,  - x$ 
        \end{itemize}
        \item Finally we know that  $ \left| a +  b \right|  $ is equal to the max of $ a +  b, -  \left( a +  b \right)  $so no matter which one it is, we have: 
            \[
            \left| a  + b \right| \le \left| a  \right|  +  \left| b \right|
            \]
    \end{itemize}
\end{proof}

Note that the triangle equality is intuitively telling us that the shortest distance between two points is a straight line
\begin{lemma}{Absolute Value is Equal to max}{absolute_value_is_equal_to_max}
    $ \forall x \in \mathbb{R}$ 
    \[
    \left| x \right| =  \max\left(x,  - x\right)
    \]
\end{lemma}
\begin{proof}
    \begin{itemize}
        \item If $ x \ge 0$ then $ \left| x \right| = x$  and $ \max\left(x, -x\right) = x$ 
        \item If $ x < 0$ then $ \left| x \right| =   - x$ and $ \max\left(x, -x\right) = -x$ since $ x < 0 \leftrightarrow  - x > 0$ and therefore $  - x > x$ 
    \end{itemize}
\end{proof}


\section{Limits}

\begin{proposition}{Limit of Constant}{limit_of_constant}
    Let $ f\left( x \right) = \alpha \in \mathbb{R}  $ be a constant function, then
    \[
        \lim_{ x \to a } f\left( x \right) = \alpha 
    \]
\end{proposition}

\begin{proof}
\begin{itemize}
    \item Let $ \varepsilon \in  \mathbb{R} ^{ +  }  $, let $ \delta  $ be fixed as any real number, then assume that $ \forall x \in  \operatorname{ dom  }\left( f \right)  $ that $ 0 < \left| x -  a \right| < \delta  $, we will show that $ \left| f\left( x \right) -  \alpha  \right| < \varepsilon  $. 
    \item Recall that $ f\left( x \right) = \alpha  $ for all $ x \in \operatorname{ dom }\left( f \right)  $, therefore $ \left| f\left( x \right) - \alpha  \right| =  \left| \alpha - \alpha  \right| =  0 < \varepsilon   $ as needed.
\end{itemize}
\end{proof}

Notice that in the above proof that we didn't use our hypothesis in the proof of the consequent, which makes sense because no matter which interval you look in the function is constant there, thus it doesn't depend on the hypothesis at all.
\begin{definition}{Real Valued Limit}{real_valued_limit}
    Suppose $ f $ is a real valued function, then we say that the limit of $ f $ at $ a $ is $ L $ and write $ \lim_{ x \to a } f\left( x \right) = L $ when the following holds:
    \[
    \forall \varepsilon \in \mathbb{R}^{+} , \exists \delta \in \mathbb{R}^{+} \text{ such that  } \forall x \in  \operatorname{ dom  } \left( f \right), 0 < \left| x -  a \right| < \delta \Rightarrow \left| f\left( x \right) -  L \right| < \varepsilon 
    \]
\end{definition}

\begin{proposition}{Constant in Limit}{constant_in_limit}
  Assume that the following limit exists $\lim_{x\to a} f\left(x\right)$ and define $L$ to be it's value, then for any $c \in \mathbb{R}$ 
  \[
  \lim_{x \to a} \left[ c f\left(x\right) \right] = c \lim_{x\to a} f\left(x\right)
  \]
\end{proposition}

\begin{proof}
\begin{itemize}
  \item If $c = 0$ then using the fact that the limit of a constant is that constant itself, we have:
    \[
      \lim_{x \to a} \left[ 0 f\left(x\right)  \right] = \lim_{x\to a} 0 = 0 = 0 \lim_{x\to a}f\left(x\right)
    \]
  \item If $c \neq 0$, we must prove that for any $\varepsilon _{c} \in \mathbb{R} ^{ > 0}$ there exists $\delta_{c}$ such that for all $x_{c} \in \operatorname{dom }\left(f\right)$ 
    \[
    \left| x_{c}  -  a \right| \le \delta_{c} \Rightarrow \left| c f\left(x_{c}\right)  -  c L \right| \le \varepsilon_{c}
    \]
  \item Notice that if we were to let $\varepsilon$ in the orignal definition be equal to $\frac{\varepsilon _{c}}{ \left| c \right|}$ then we could multiply the equation after the implication on both sides by $ \left| c \right|$ (so that we can absorb it into the absolute value).
  \item Let $\varepsilon_{c} \in \mathbb{R} ^{> 0}$ since the original limit holds for any epsilon, bind $\varepsilon$ to $\frac{\varepsilon _{c}}{ \left| c \right|}$ and we get $\delta$ such that for all $x \in \operatorname{dom }\left(f\right)$, the following holds:
    \[
      \quad \left| x  -  a \right| \le \delta \Rightarrow \left| f\left(x\right)  -  L \right| \le \frac{\varepsilon _{c}}{ \left| c \right|} \tag{$\alpha$}
    \]
  \item Take $\delta_{c} = \delta$, let $x_{c} \in \operatorname{dom }\left(f\right)$ and bind $x$ in the original definition to $x_{c}$, and assume that $ \left| x_{c}  -  a \right|\le \delta_{c}$, because of our choice for $\delta_{c}$ we satisfy $\alpha$'s hypothesis with $x$ replaced by $x _{c}$ and we get
    \[
    \left| f\left(x_{c}\right)  -  L \right| \le \frac{\varepsilon_{c}}{c} \Leftrightarrow \left| c \right| \left| f\left(x\right)  -  L \right| \le \varepsilon_{c}
    \]
    \item Since for any $a, b \in \mathbb{R}$ we have $ \left|  a b \right|= \left| a \right| \left|  b \right|$ we can conclude with distributivity in $ \mathbb{R}$ that
      \[
      \left| c f\left(x _{c}\right)  -  c L \right| \le \varepsilon_{c}
      \]
     As required.
\end{itemize}
\end{proof}

\begin{proposition}{Sum of Limits}{sum_of_limits}
If $\lim _{x \rightarrow a} f(x)=L$ and $\lim _{x \rightarrow a} g(x)=M$ both exist then
$$
\lim _{x \rightarrow a}[f(x)+g(x)]=L+M
$$
\end{proposition}
\begin{proof}
    \begin{itemize}
        \item Suppose antecedent holds, so we have existsance of the limits $\lim _{x \rightarrow a} f(x)=L$ and $\lim _{x \rightarrow a} g(x)=M$, now we'd like to show that the limit $\lim _{x \rightarrow a}[f(x)+g(x)]$ equals $L+M$ 
        \item Let $ \varepsilon \in \mathbb{R} ^{ +  }$, now from the fact that the two prior limits exist we can substitute in $ \frac{\varepsilon}{2}  $ into them, to get $ \delta _{ L }  $ and $ \delta _{ M }  $ respectively, such that 
        \[
        \forall x \in  \operatorname{ dom  }\left( f \right) , 0 < \left| x -  a \right| < \delta _{ L }  \Rightarrow \left| f\left( x \right) - L \right| < \frac{\varepsilon}{2}\\
        \]
        and
        \[
        \forall x \in  \operatorname{ dom  }\left( g \right) , 0 < \left| x -  a \right| < \delta _{ M }  \Rightarrow \left| g\left( x \right) - M \right| < \frac{\varepsilon}{2}\\
        \]
        \item Take $ \delta =  \min\left( \delta _{ L }  , \delta _{ M }  \right)  $, so that we may utilize the above inequalities as we assume that $ \forall x \in  \operatorname{ dom  }\left( f +  g \right)  $ that $ 0 < \left| x -  a \right| < \delta  $, and proceed to show that $ \left| \left( f\left( x \right) +  g\left( x \right)  \right) -  \left( L +  M \right) \right| < \varepsilon  $, since our $ x $ is bounded by the min of the two deltas from the other limits, we utilize the inequalities in tandem with the triangle equality to obtain: 
        \[
        \left| \left( f\left( x \right)  +  g\left( x \right)  \right) -  \left( M +  L \right) \right| =  \left| \left( f\left( x \right) -  L \right) +  \left( g\left( x \right) -  M \right) \right| \le \left| f\left( x \right) -  L \right| + \left| g\left( x \right) -  M \right| < \frac{\varepsilon }{2} +  \frac{\varepsilon }{2} =  \varepsilon 
        \]
    \end{itemize}
\end{proof}


\newpage

\section{Differentiation}

\begin{theorem}{Chain Rule}{chain_rule}
    Given two functions $ f $ and $ g $ where $ g $ is differentiable at the point $ \overline{x}  $  and $ f $ differentiable at the point $ \overline{y} =  g\left( \overline{x}  \right)  $ then
    \[
    \left( f \circ g \right) ^{ \prime }  \left( \overline{x}  \right) =  f ^{ \prime } \left( g\left( \overline{x}  \right)  \right) \cdot g\left( \overline{x}  \right) 
    \]
\end{theorem}

\begin{proof}
    \begin{itemize}
        \item We define the following two new functions
            \[
            v\left( h \right) =  \frac{g\left( \overline{x} +  h \right) - g\left( \overline{x} \right) }{h} -  g ^{ \prime } \left( \overline{x} \right)  \text{ and } w\left( k \right) =  \frac{f\left( \overline{y}  +  k \right) -  f\left( \overline{y}  \right) }{k} -  f ^{ \prime } \left( \overline{y}  \right) 
            \]
        \item[$ \alpha: $] Note that $ \lim_{ h \to 0 } v\left( h \right) = 0 $ and $ \lim_{ k \to 0 } w\left( k \right)  = 0 $, and that we can re-arrange the above to:

            \[
            \left( v\left( h \right) +  g ^{ \prime } \left( \overline{x}  \right)  \right) \cdot h +  g\left( \overline{x}  \right) = g\left( \overline{x} + h \right) \text{ and } \left( w\left( k \right) +  f ^{ \prime } \left( \overline{y}  \right)  \right) \cdot k +  f\left( \overline{y}  \right) = f\left( \overline{y} + k \right) 
            \]
        \item Now it follows that
            \begin{align*}
                f\left( g\left( \overline{x} + h \right)  \right) &= f\left( \left( v\left( h \right) +  g ^{ \prime } \left( \overline{x}  \right)  \right) \cdot h +  g\left( \overline{x}  \right) \right) \\
&= f\left(g\left( \overline{x}  \right) + \left(    v\left( h \right) +  g ^{ \prime } \left( \overline{x}  \right)  \right) \cdot h  \right) \\
&= f\left( \overline{y}  + \left(    v\left( h \right) +  g ^{ \prime } \left( \overline{x}  \right)  \right) \cdot h  \right) \\
&= \left( w\left( \left(    v\left( h \right) +  g ^{ \prime } \left( \overline{x}  \right)  \right) \cdot h \right) +  f ^{ \prime } \left( \overline{y}  \right)  \right) \cdot \left(    v\left( h \right) +  g ^{ \prime } \left( \overline{x}  \right)  \right) \cdot h +  f\left( \overline{y}  \right)
            \end{align*}
        \item Now recall that we are interested in $ \left( f \circ g \right)' \left( \overline{x}  \right) $, so we're going to have to look at:
            \[
            \lim_{ h \to 0 } \frac{f\left( g\left( \overline{x} +  h \right)  \right)-  f\left( g\left( \overline{x}  \right)  \right)   }{h}
            \]
        \item Notice that the term on the left of the sum in the numerator is something we already know, and therefore it becomes:
        \[
        \lim_{ h \to 0 } \frac{\left( w\left( \left(    v\left( h \right) +  g ^{ \prime } \left( \overline{x}  \right)  \right) \cdot h \right) +  f ^{ \prime } \left( \overline{y}  \right)  \right) \cdot \left(    v\left( h \right) +  g ^{ \prime } \left( \overline{x}  \right)  \right) \cdot h +  f\left( \overline{y}  \right)-  f\left( g\left( \overline{x}  \right)  \right)   }{h}
        \]
        \item By noting that $ f\left( \overline{y}  \right) = f\left( g\left( \overline{x}  \right)  \right)  $, then cancelling out the $ h $ we can see that the above simplifies to 
        \[
        \lim_{ h \to 0 } \left( w\left( \left(    v\left( h \right) +  g ^{ \prime } \left( \overline{x}  \right)  \right) \cdot h \right) +  f ^{ \prime } \left( \overline{y}  \right)  \right) \cdot \left(    v\left( h \right) +  g ^{ \prime } \left( \overline{x}  \right)  \right) 
        \]
        \item By $ \alpha  $ on $ v $ and $ w $ with $ k =  g\left( \overline{x}  \right) \cdot h  $, we have that the above limit is:
        \[
        \lim_{ h \to 0 } \left( w\left( \left( g ^{ \prime } \left( \overline{x}  \right)  \right) \cdot h \right) +  f ^{ \prime } \left( \overline{y}  \right)  \right) \cdot \left(     g ^{ \prime } \left( \overline{x}  \right)  \right)  =  \lim_{ h \to 0 } f ^{ \prime } \left( \overline{y}  \right) \cdot g ^{ \prime } \left( \overline{x}  \right) = \boxed{ f ^{  \prime } \left( g\left( \overline{x}  \right)  \right)  \cdot  g ^{ \prime } \left( \overline{x}  \right) }
        \]
    \end{itemize}
\end{proof}


The chain rule is sometimes written as $ \frac{df }{dx} \left( x \right) = \frac{df}{dg} \left( x \right) \frac{dg}{dx} \left( x \right)  $, but notice that this extends liebniz notation as we only know that $ \frac{df}{dx} = f ^{ \prime  } \left( x \right)   $, and it is not defined for when we take the derivative with respect to another function. This motivates the definition
\[
\frac{df}{dg} \left( x \right) =  \frac{df}{dx} \left( g\left( x \right)  \right)
\]
\begin{proposition}{Little O and Differentiability Equivalence}{little_o_and_differentiability_equivalence}
  A function $f$ is differentiable at the point $\overline{x}$ if and only if there is a number $ \alpha \in \mathbb{R}$ for any $h \in \mathbb{R}$ we have 
  \[
  f\left(\overline{x}  +  h\right) = f\left(\overline{x}\right)  +  h \alpha  +  l\left(h\right)
  \]
  Where $l \in o\left(h\right) $ 

\end{proposition}

\begin{proof}
    \begin{itemize}
      \item $\Rightarrow$ 
        \begin{itemize}
          \item Assume that $f$ is differentiable, therefore we have
              \[
              f ^{ \prime } \left( \overline{x} \right) =  \lim_{ h \to 0 } \frac{f\left( \overline{x} +  h \right) -  f\left( x \right) }{h}
              \]
            \item Equivalently:
              \[
               \lim_{ h \to 0 } \frac{f\left( \overline{x} +  h \right) -  f\left( x \right) }{h} -  f ^{ \prime } \left( \overline{x} \right) = 0
              \]
            \item Since $ \lim_{ h \to 0 } f ^{ \prime } \left( \overline{x} \right) = f ^{ \prime } \left( \overline{x} \right)  $ as it's a constant, we may use the limit law to deduce:
            \[
            \lim_{ h \to 0 } \left( \frac{f\left( \overline{x} + h \right) - f\left( x \right) }{h} - f ^{ \prime } \left( \overline{x} \right) \right) = 0 \Leftrightarrow \lim_{ h \to 0 } \frac{f\left( \overline{x} + h \right) - f\left( \overline{x} \right) - h f ^{ \prime } \left( \overline{x} \right) }{h}= 0
            \]
            \item That is to say that $ f\left( \overline{x} + h \right) - f\left( \overline{x} \right) - h f ^{ \prime } \left( \overline{x} \right) \in o\left( h \right)  $, set $ l\left( h \right) = f\left( \overline{x} + h \right) - f\left( \overline{x} \right) - h f ^{ \prime } \left( \overline{x} \right)  $ then we have that
            \[
            f\left( \overline{x} + h \right) =  f\left( \overline{x} \right) +  h f ^{ \prime } \left( \overline{x} \right) +  l\left( h \right) 
            \]
            where we've taken $ \alpha = f ^{ \prime } \left( \overline{x} \right)  $ 
        \end{itemize}
      \item $\Leftarrow$ 
        \begin{itemize}
          \item Assume that there is a number $\alpha \in \mathbb{R}$ and $l \in o\left(h\right)$, so that for any $h \in \mathbb{R}$  we have:
            \[
            f\left(\overline{x}  +  h\right) = f\left(\overline{x}\right)  +  h \alpha+  l\left(h\right) \Leftrightarrow l\left(h\right) = f\left(\overline{x}  +  h\right)  -  f\left(\overline{x}\right)  -  h \alpha
            \]
            thus $f\left(\overline{x}  +  h\right)  -  f\left(\overline{x}\right)  -  h \alpha \in o\left(h\right)$.
            \item So then by definition of little-o, we get:
            \[
           \lim_{h\to0} \frac{f\left(\overline{x}  +  h\right)  -  f\left(\overline{x}\right)  -  h \alpha }{h} = 0
            \]
            \item Since $ \lim_{ h \to 0 } \alpha = \alpha $ we may add it to both sides using the limit law on the left to obtain:
            \[
            \lim_{ h \to 0 } \left( \frac{f\left( \overline{x} +  h \right) -  f\left( \overline{x} \right) -  h \alpha }{h} + \alpha  \right)= \alpha \Leftrightarrow \lim_{ h \to 0 } \frac{f\left( \overline{x} + h \right) -  f\left( \overline{x} \right) }{h} = \alpha 
            \]
            Meaning that $ \alpha = f ^{ \prime } \left( \overline{x} \right)  $ and that $ f $ is differentiable
        \end{itemize}
    \end{itemize}
\end{proof}

Notice that the number $ \alpha $ is equal to a limit and that limits are unique, therefore the solution $ \alpha =  f ^{ \prime } \left( \overline{x} \right)  $ is the unique solution to the above proposition. Also since this it the unique solution, we know that there is no other real besides $ f ^{ \prime } \left( \overline{x} \right)  $ where this holds, therefore $ f ^{ \prime } \left( \overline{x} \right)  $ is the best linear approximation to $ f $ at the point $ \overline{x} $. 

Finally notice that if you solve for $ E\left( h \right) $ and then recall that $ E\left( h \right) \in o\left( h \right)  $ we have:
\[
\lim_{ h \to 0 } \frac{ \overbracket{\overbracket{f\left( \overline{x} +  h \right)}^{~\text{actual value}~ } -  \overbracket{\left( f\left( \overline{x}  \right)  +  h f ^{ \prime  } \left( \overline{x}  \right) \right)}^{~\text{linear estimation}~  }}^{~\text{error}~  } }{h}= 0
\]

the error is going to zero at a rate which is faster than linear, as if the error were decreasing at a linear rate the limit would evaluate to some non-zero constant.

\newpage
\section{Sequences}


\begin{definition}{Bounded Sequence}{bounded_sequence}
A sequence $ \left\{ x _{ n }  \right\}  $ is said to be bounded if $ \exists M \in \mathbb{R}   $ such that $ \left| x _{ n }  \right| \le M $ for all $ n \in  \mathbb{N}  $ 
\end{definition}



\section{Multi Variable}

\begin{definition}{Directional Derivative}{directional_derivative}
    Let $ U \subseteq \mathbb{R} ^{ n } $ be open, $f : U \to \mathbb{R} $ and $ \vec{a} \in  U $ and $ \vec{v} \in  \mathbb{R} ^{ n }  $ 
    \[
        \partial_{\vec{v}} f \left( \vec{a}  \right) \stackrel{\mathtt{D}}{=} \lim_{ t \to 0 \in \mathbb{R}  } \frac{f\left( \vec{a} + t \vec{v}  \right) - f\left( \vec{a}  \right) }{t}
    \]
\end{definition}

