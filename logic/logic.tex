\chapter{First Order Logic}

\section{Languages}

\begin{definition}{First-Order Language}{first_order_language}
	A first order language $\mathcal{L}$ is an infinite collection of distinct symbols, no one of which is properly contained in another, sparated into the following cateogories:
	\begin{enumerate}
		\item Parentheses: (,).
		\item Connectives: $\lor, \neg$.
		\item Quantifier: $\forall$.
		\item Variables, one for each positive integer n: $v_{1}, v_{2}, \ldots, v_{n}, \ldots$ The set of variable symbols will be denoted Vars.
		\item Equality symbol: $=$.
		\item Constant symbols: Some set of zero or more symbols.
		\item Function symbols: For each positive integer $n$, some set of zero or more $n$-ary function symbols.
		\item Relation symbols: For each positive integer $n$, some set of zero or more $n$-ary relation symbols.
	\end{enumerate}
\end{definition}


\begin{definition}{A Term's Variable Replaced by a Term}{a_term's_variable_replaced_by_a_term}
    Suppose that $u$ is a term, $x$ is a variable, and $t$ is a term. We define the term $u_{t}^{x}$ (read ``$u$ with $x$ replaced by $\left.t "\right)$ as follows:
    \begin{enumerate}
        \item If $u$ is a variable not equal to $x$, then $u_{t}^{x}$ is $u$.
        \item If $u$ is $x$, then $u_{t}^{x}$ is $t$.
        \item If $u$ is a constant symbol, then $u_{t}^{x}$ is $u$.
        \item If $u: \equiv f u_{1} u_{2} \ldots u_{n}$, where $f$ is an $n$-ary function symbol and the $u_{i}$ are terms, then $u_{t}^{x}$ is $f\left(u_{1}\right)_{t}^{x}\left(u_{2}\right)_{t}^{x} \ldots\left(u_{n}\right)_{t}^{x}$
    \end{enumerate}
\end{definition}


\begin{itemize}
    \item  In the fourth clause of the definition above and in the first two clauses of the next definition, the parentheses are not really there. Because $u_{1}{ }_{t}^{x}$ is hard to read so the parentheses have been added in the interest of readability.
    \item If we let $t$ be $g(c)$ and we let $u$ be $f(x, y)+h(z, x, g(x))$, then $u_{t}^{x}$ is
        \[
        f(g(c), y)+h(z, g(c), g(g(c)))
        \]
\end{itemize}


\begin{definition}{A Term is Substitutable for a Variable}{a_term_is_substitutable_for_a_variable}
Suppose that \(\phi\) is an \(\mathcal{L}\)-formula, \(t\) is a term, and \(x\) is a variable. We say that \(t\) is substitutable for \(x\) in \(\phi\) if
\begin{enumerate}
    \item \(\phi\) is atomic, or
    \item \(\phi: \equiv \neg(\alpha)\) and \(t\) is substitutable for \(x\) in \(\alpha\), or
    \item \(\phi: \equiv(\alpha \vee \beta)\) and \(t\) is substitutable for \(x\) in both \(\alpha\) and \(\beta\), or
    \item \(\phi: \equiv(\forall y)(\alpha)\) and either
    \begin{itemize}
        \item \(x\) is not free in \(\phi\), or
        \item \(y\) does not occur in \(t\) and \(t\) is substitutable for \(x\) in \(\alpha\).
    \end{itemize}
\end{enumerate}
\end{definition}


To understand the motivation behind the fourth clause, consider the formula:
\[
\phi :\equiv \left( \forall x \right) \left( \forall y \right) x = y
\]

Then one might want to say that $ \left( \left( \forall y \right) x = y \right) _{ t }^{ x }   $  where $ t $ is any term

\begin{definition}{Free Variable in a Formula}{free_variable_in_a_formula}
    Suppose that $v$ is a variable and $\phi$ is a formula. We will say that $v$ is free in $\phi$ if
    \begin{enumerate}
        \item $\phi$ is atomic and $v$ occurs in (is a symbol in) $\phi$, or                               
        \item $\phi: \equiv(\neg \alpha)$ and $v$ is free in $\alpha$, or
        \item $\phi: \equiv(\alpha \vee \beta)$ and $v$ is free in at least one of $\alpha$ or $\beta$, or
        \item $\phi: \equiv(\forall u)(\alpha)$ and $v$ is not $u$ and $v$ is free in $\alpha$.
    \end{enumerate}
\end{definition}


\subsubsection*{Examples}
\begin{itemize}
    \item Thus, if we look at the formula
    $$
    \forall v_{2} \neg\left(\forall v_{3}\right)\left(v_{1}=S\left(v_{2}\right) \vee v_{3}=v_{2}\right)
    $$
    the variable $v_{1}$ is free whereas the variables $v_{2}$ and $v_{3}$ are not free. 
    \item A slightly more complicated example is
        \[
        \left(\forall v_{1} \forall v_{2}\left(v_{1}+v_{2}=0\right)\right) \vee v_{1}=S(0)
        \]
        \begin{itemize}
            \item In this formula, $v_{1}$ is free whereas $v_{2}$ is not free. Especially when a formula is presented informally, you must be careful about the scope of the quantifiers and the placement of parentheses.
        \end{itemize}
\end{itemize}
\subsubsection*{Notes}
\begin{itemize}
    \item We will have occasion to use the informal notation $\forall x \phi(x)$. This will mean that $\phi$ is a formula and $x$ is among the free variables of $\phi$. 
    \item If we then write $\phi(t)$, where $t$ is an $\mathcal{L}$-term, that will denote the formula obtained by taking $\phi$ and replacing each occurrence of the variable $x$ with the term $t$. 
\end{itemize}


\section{Deductions}
\begin{definition}{Deduction of a Formula}{deduction_of_a_formula}
Suppose that $\Sigma$ is a collection of $\mathcal{L}$-formulas and $D$ is a finite sequence $\left(\phi_{1}, \phi_{2}, \ldots, \phi_{n}\right)$ of $\mathcal{L}$-formulas. We will say that $D$ is a deduction from $\Sigma$ if for each $i, 1 \leq i \leq n$, either
\begin{enumerate}
    \item $\phi_{i} \in \Lambda\left(\phi_{i}\right.$ is a logical axiom), or
    \item $\phi_{i} \in \Sigma\left(\phi_{i}\right.$ is a nonlogical axiom), or
    \item There is a rule of inference $\left(\Gamma, \phi_{i}\right)$ such that $\Gamma \subseteq\left\{\phi_{1}, \phi_{2}, \ldots, \phi_{i-1}\right\}$.
\end{enumerate}
If there is a deduction from $\Sigma$, the last line of which is the formula $\phi$, we will call this a deduction from $\Sigma$ of $\phi$, and write $\Sigma \vdash \phi$.
\end{definition}


\begin{definition}{Logical Axioms}{logical_axioms}
 Given a first order language $ \mathcal{ L }   $ the set $\Lambda$ of logical axioms is the collection of all $ \mathcal{ L }-$formulas that fall into one of the following categories:
\begin{gather*}
    x=x  \text { for each variable } x \\
    \left[\left(x_{1}=y_{1}\right) \wedge\left(x_{2}=y_{2}\right) \wedge \cdots \wedge\left(x_{n}=y_{n}\right)\right] \rightarrow \left(f\left(x_{1}, x_{2}, \ldots, x_{n}\right)=f\left(y_{1}, y_{2}, \ldots, y_{n}\right)\right) \\
    \left[\left(x_{1}=y_{1}\right) \wedge\left(x_{2}=y_{2}\right) \wedge \cdots \wedge\left(x_{n}=y_{n}\right)\right] \rightarrow \left(R\left(x_{1}, x_{2}, \ldots, x_{n}\right) \rightarrow R\left(y_{1}, y_{2}, \ldots, y_{n}\right)\right)\\
    \left( \forall x \phi  \right) \rightarrow \phi _{ t }^{ x } ~\text{if $ t $ is substitutable for $ x $ in $ \phi  $ }~  \\
    \phi _{ t }^{ x } \to \left( \exists x \phi \right) ~\text{if $ t $ is substitutable for $ x $ in $ \phi  $ }~  
\end{gather*}
To refer to them easily we label them by  moving down the above list E1, E2, E3, Q1, Q2 
\end{definition}


\begin{lemma}{Universal connection to Variable Assignment Function}{forall vaf}
    \[
    \Sigma \vdash \theta \text { if and only if } \Sigma \vdash \forall x \theta
    \]
\end{lemma}

%Note this lemma might seem quite strange, but note it actually makes sense, %todo{finish why}

\begin{proof}
    \begin{itemize}
        \item $ \Rightarrow $ 
        \begin{itemize}
            \item Suppose that $ \Sigma \vdash \theta$, therefore we have a deduction $ \mathcal{D}$ of $ \theta$, then the proof
                \begin{gather*}
                    \mathcal{D}\\
                    \left[ \left( \forall y \left( y =  y \right) \right) \land  \neg \left( \forall y \left( y =  y \right) \right) \right] \rightarrow \theta \tag{taut. PC}\\
                    \left[ \left( \forall y \left( y =  y \right) \right) \land  \neg \left( \forall y \left( y =  y \right) \right) \right] \rightarrow \left(  \forall  x \right)\theta \tag{QR}\\
                    \left(  \forall x \right) \theta \tag{PC}
                \end{gather*}
        \end{itemize}
        \item $ \Leftarrow $
        \begin{itemize}
            \item Suppose that $ \Sigma \vdash \forall x \theta$, so we have a deduction of it, call it  $\mathcal{D}$, then the following deduction suffices
            \begin{gather*}
               \mathcal{D} \\
               \forall x \theta\\
               \forall x \theta \rightarrow \theta _{x}^{x}\\
                \theta _{x}^{x}
            \end{gather*}
        \end{itemize}
    \end{itemize}
\end{proof}


\section{Completeness}

\begin{definition}{Consistent Set of L Formulas}{consistent_set_of_l_formulas}
Let $\Sigma$ be a set of $\mathcal{L}$-formulas. We will say that $\Sigma$ is inconsistent if there is a deduction from $\Sigma$ of $[(\forall x) x=x] \wedge \neg[(\forall x) x=x]$. We say that $\Sigma$ is consistent if it is not inconsistent.
\end{definition}


\begin{proposition}{Contradiction has no Model}{contradiction_has_no_model}
    \begin{center}
        There is no $\mathcal{L}$-Structure $ \mathfrak{ A }   $ such that $ \mathfrak{ A } \models \bot $ 
    \end{center}
\end{proposition}
\begin{proof}
    Suppose there was a model of $ \bot $, that would mean that we have an $\mathcal{L}$-Structure $ \mathfrak{ A }     $ such that $ \mathfrak{ A } \models \bot  $. Recall that $ \bot :\equiv \left[ \left( \forall x \right) x =  x \right] \land \neg \left[ \left( \forall x \right)x =  x \right] $, so then we would have to have $ \mathfrak{ A } \models  \left( \forall x \right)x =  x   $ and also not have $ \mathfrak{ A } \models \left( \forall x \right) x =  x  $ which is a contradiction.
\end{proof}


\begin{lemma}{Constant Extension still Consistent}{constant_extension_still_consistent}
If $\Sigma$ is a consistent set of $\mathcal{L}$-sentences and $\mathcal{L}^{\prime}$ is an extension by constants of $\mathcal{L}$, then $\Sigma$ is consistent when viewed as a set of $\mathcal{L}^{\prime}$-sentences.
\end{lemma}
\begin{proof}
    \begin{itemize}
        \item Suppose for the sake of contradiction that $ \Sigma  $ is not consistent when viewed as a set of $\mathcal{L}^{\prime}$-sentences, so $ \Sigma \vdash \bot $, considering the set of all deductions of $ \bot  $ from $ \Sigma  $ we may find a deduction $ \mathcal{ D }   $  which uses the least number of the newly added constants by the well ordering principle, let this number be $ n \in  \mathbb{N} $ and that $ n > 0 $ or else we would have a deduction from $ \mathcal{ L }   $ of $ \bot  $ which would be a contradiction.
        \item Let $ v $ be a variable that isn't used in $ \mathcal{ D }   $ and let $ c $ be one of the newly added constants which is used in $ \mathcal{ D }   $ and let $ \mathcal{ D } _{ v }   $ be created where for each line $ \phi \in  \mathcal{ D }  $ we create $ \phi _{ v }  $ by replacing all occurances of $ c $ in $ \phi  $ with $ v $, and note that the last line of $ \mathcal{ D } _{ v }   $ is still $ \bot  $, at this point we don't know if $ \mathcal{ D } _{ v }   $ is a deduction, so we have to check that it is.
        \item If $ \phi  $ is an equality axiom or an element of $ \Sigma $ then $ \phi _{ v } :\equiv \phi  $ because equality axioms only contain variables and not constants, also $ \Sigma  $ is a set of $ \mathcal{ L }   $ sentences and so it can't contain any of the new constants, and so $ \phi _{ v }  $ is a valid step in the deduction since it is still an equality axiom or an element of $ \Sigma  $ 
        \item If $ \phi  $ is $ \left( \forall x \right) \theta \rightarrow \theta _{ t }^{ x }  $ then $ \phi _{ v }  $ is $ \left( \forall x \right)\theta _{ v } \rightarrow \left( \theta _{ v }  \right) _{ t _{ v }  }^{ x  }  $, to see why we use $ t _{ v }  $ as well as $ \theta _{ v }  $ try $ \theta :\equiv c =  x $ and $ t :\equiv   c + 3 $ 
    \end{itemize}
\end{proof}


Notice that if we write $ \Sigma \models \bot $ it means that for any $\mathcal{L}$-Structure $ \mathfrak{ A }   $ if $ \mathfrak{ A } \models \Sigma   $ then $ \mathfrak{ A } \models \bot  $ by the above discussion that forces $ \mathfrak{ A } \models \Sigma   $ to be false, and therefore $ \Sigma  $ has no model.

\begin{theorem}{Completeness Theorem}{completeness}
Suppose that $\Sigma$ is a set of $\mathcal{L}$-formulas, where $ \mathcal{L}$ is a countable langauge  and $\phi$ is an $\mathcal{L}$-formula. If $\Sigma \models \phi$, then $\Sigma \vdash \phi$.

\section*{Setup}

\begin{itemize}
    \item We start by assuming that $ \Sigma \models \phi$, we must show that $ \Sigma \vdash \phi$.
    \item If $ \phi$ is not a sentence then we can always prove $ \phi'$ which is the same as $ \phi$ with all of it's variables bound
    \begin{itemize}
        \item We can do that by appending $ \left( \forall  x _{f}  \right)$ where each $ x_{f}$  is a free varaible of $ \phi$ to the front of $ \phi$ 
    \end{itemize}
\item Therefore we will prove it for all sentences $ \phi$ %\todo[inline]{justify why this is equivalent}
\end{itemize}

\end{theorem}


\section{Incompleteness}

\begin{definition}{Representable Set}{representable_set}
A set $A \subseteq \mathbb{N}^{k}$ is said to be representable (in $N$) if there is an $\mathcal{L}_{N T}$-formula $\phi(\underline{x})$ such that
$$
\begin{array}{ll}
\forall a \in A & N \vdash \phi(\bar{a}) \\
\forall \underline{b} \notin A & N \vdash \neg \phi(\bar{b})
\end{array}
$$
In this case we will say that the formula $\phi$ represents the set $A$.
\end{definition}

\begin{definition}{Weakly Representable Set}{weakly_representable_set}
A set $A \subseteq \mathbb{N}^{k}$ is said to be weakly representable (in $N$ ) if there is an $\mathcal{L}_{N T}$-formula $\phi(x)$ such that
$$
\begin{array}{ll}
\forall a \in A & N \vdash \phi(\bar{a}) \\
\forall b \notin A & N \nvdash \phi(\bar{b})
\end{array}
$$
In this case we will say that the formula $\phi$ weakly represents the set $A$.
\end{definition}

\begin{definition}{Total Function}{total_function}
Suppose that $A \subseteq \mathbb{N}^{k}$ and suppose that $f: A \rightarrow \mathbb{N}$. If $A=\mathbb{N}^{k}$ we will say that $f$ is a total function
\end{definition}

\begin{definition}{Partial Function}{partial_function}
Suppose that $A \subsetneq \mathbb{N}^{k}$ and suppose that $f: A \rightarrow \mathbb{N}$, then we will call $f$ a partial function.
\end{definition}

\begin{definition}{Representable Function}{representable_function}
    Suppose that $f: \mathbb{N}^{k} \rightarrow \mathbb{N}$ is a total function. We will say that $f$ is a representable function (in $N$ ) if there is an $\mathcal{L}_{N T}$ formula $\phi\left(x_{1}, \ldots, x_{k+1}\right)$ such that, for all $a_{1}, a_{2}, \ldots a_{k+1} \in \mathbb{N}$
    \begin{itemize}
        \item If $f\left(a_{1}, \ldots, a_{k}\right)=a_{k+1}$, then $N \vdash \phi\left(\overline{a_{1}}, \ldots, \overline{a_{k+1}}\right)$
        \item If $f\left(a_{1}, \ldots, a_{k}\right) \neq a_{k+1}$, then $N \vdash \neg \phi\left(\overline{a_{1}}, \ldots, \overline{a_{k+1}}\right)$
    \end{itemize}
\end{definition}

\begin{definition}{Definable Set}{definable_set}
We will say that a set $A \subseteq \mathbb{N}^{k}$ is definable if there is a formula $\phi(x)$ such that
$$
\begin{array}{ll}
\forall a \in A & \mathfrak{N} \models \phi(\bar{a}) \\
\forall \underline{\downarrow b} \notin A & \mathfrak{N} \models \neg \phi(\bar{b})
\end{array}
$$
In this case, we will say that $\phi$ defines the set $A$.
\end{definition}

\begin{corollary}
    If $ A \subseteq \mathbb{N} ^{ k }  $ is definable by a $ \Delta  $ formula, then it is representable
\end{corollary}

