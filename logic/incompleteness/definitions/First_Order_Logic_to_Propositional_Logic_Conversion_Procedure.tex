\begin{definition}{First Order Logic to Propositional Logic Conversion Procedure}
Given $\beta$, an $\mathcal{L}$-formula of first-order logic, here is a procedure that will convert $\beta$ to a formula $\beta_{P}$ of propositional logic corresponding to $\beta$
\begin{enumerate}
    \item Find all subformulas of $\beta$ of the form $\forall x \alpha$ that are not in the scope of another quantifier. Replace them with propositional variables in a systematic fashion. This means that if $\forall y Q(y, c)$ appears twice in $\beta$, it is replaced by the same letter both times, and distinct subformulas are replaced with distinct letters.
    \item Find all atomic formulas that remain, and replace them systematically with new propositional variables.
    \item At this point, $\beta$ will have been replaced with a propositional formula $\beta_{P}$
\end{enumerate}

\begin{ex}
    \begin{itemize}
        \item Suppose that we look at the $\mathcal{L}$-formula
$$
(\forall x P(x) \wedge Q(c, z)) \rightarrow(Q(c, z) \vee \forall x P(x))
$$
For the first step of the procedure above, we replace the quantified subformulas with the propositional letter $B$ :
$$
(B \wedge Q(c, z)) \rightarrow(Q(c, z) \vee B)
$$
To finish the transformation to a propositional formula, replace the atomic formula with a propositional letter:
$$
(B \wedge A) \rightarrow(A \vee B)
$$
    \end{itemize}
\end{ex}
\end{definition}
