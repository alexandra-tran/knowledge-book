\documentclass{standalone}
\usepackage{tikz,lmodern,amssymb}
\usepackage{knowledge}

\begin{document}

\begin{deduction*}
{Complex Multiplicative Inverse Formula}

Let \(z = a + bi \in \mathbb{C} \setminus \{0\}\), then
\[
z^{-1} = \frac{a - bi}{a^{2} + b^{2}} = \frac{\overline{z}}{\left| z \right|^{2}}
\]

\begin{pf}
    \begin{itemize}
        \item \(z^{-1}\) is the element \(x + yi \in \mathbb{Z}\) which satisfies \(z \left(x + yi\right) = 1 = 1 + 0i\),
        \[
        \left(a + bi\right)\left(x + yi\right) = ax - by + i \left(ay + bx\right)
        \]
        \item We require that \(ax - by = 1\) and \(ay + bx = 0\), naively we can take \(y = - b\) and \(x = a\) to solve the second equation, but then the first equation would tell us that \(a^{2} + b^{2} = 1\) which is only true when \(z\) lies on the unit circle, otherwise we will need to take \(y = \frac{-b}{\left(a^{2} + b^{2}\right)}\) and \(x = \frac{a}{\left(a^{2} + b^{2}\right)}\), subbing this into our first equation gives \(\frac{a^{2} + b^{2}}{a^{2} + b^{2}} = 1\) which is certainly true.
        \item Therefore we can see that
            \[
            z^{-1} = \frac{a - bi}{a^{2} + b^{2}}
            \]
    \end{itemize}
    \subsubsection*{Remark}
    \begin{itemize}
        \item Also note when we have the main equality we get the following nice equation
        \[
        z^{-1} = \frac{\overline{z}}{\left| z \right|^{2}} \Leftrightarrow \left| z \right|^{2} = \overline{z} \cdot z \Leftrightarrow \left| \overline{z} \right| \cdot \left| z \right| \stackrel{\text{pretty}}{ = } \overline{z} \cdot z
        \]
    \end{itemize}

\end{pf}

\end{deduction*}

\end{document}
