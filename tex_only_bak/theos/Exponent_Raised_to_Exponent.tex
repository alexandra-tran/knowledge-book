\documentclass{standalone}
\usepackage{tikz,lmodern,amssymb}
\usepackage{knowledge}

\usepackage{algpseudocode}% http://ctan.org/pkg/algorithmicx

\begin{document}

\begin{deduction*}{Exponent Raised to Exponent}
    for $ x \in \mathbb{R} ^{ \ge 0}$ and $ a, b \in \mathbb{R} \setminus \left\{ 0 \right\}$, then 
    \[
    \left( x ^{a} \right) ^{ b} =  x ^{ \left( a  \cdot b \right)}
    \]
    \begin{pf}
        \begin{itemize}
            \item From the definition of the exponent, the left hand side and \kref{https://gitlab.com/cuppajoeman/knowledge-data/-/blob/master/Exponent_Addition-MjL4gCoh3dRCLhYW7KV.pdf}{exponent addition}, we have: 
                \[
                \underbracket{x ^{a}  x ^{a}  \cdot \dotsm x ^{a} x ^{a}}_{ b \text{ times}}  =  x ^{ \underbracket{a  +  a  +  \dotsb  +  a  +  a}_{b \text{ times}}} =  x ^{ a  \cdot  b}
                \]
        \end{itemize}
    \end{pf}
    \subsubsection*{Notes}
    \begin{itemize}
        \item To see why $ x $ must be non-negative, we can consider $ x =  -1$, then 
            \[
                \left( -1 \right) ^{1} =  \left( -1 \right) ^{ 2  \cdot  \frac{1}{2}} =  \left( -1 ^{2} \right) ^{\frac{1}{2}} =  1
            \]
            but $ 1 \neq  -1$
    \end{itemize}
\end{deduction*}

\end{document}


