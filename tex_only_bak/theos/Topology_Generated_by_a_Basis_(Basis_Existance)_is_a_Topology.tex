\documentclass{standalone}
\usepackage{tikz,lmodern,amssymb}
\usepackage{knowledge}

\usepackage{algpseudocode}

\begin{document}

\begin{deduction*}
{Topology Generated by a Basis (Basis Existance) is a Topology}
    \begin{center}
        We claim that \(\mathcal{T}_{B}\) is a topology
    \end{center}
    \begin{pf}
        \begin{itemize}
            \item Vacuously we see that \(\varnothing \in \mathcal{T} _{B}\), and \(X \in \mathcal{T} _{B}\) by the first clause of the definition of a basis
            \item We will show that \(\bigcap_{i = 1}^{n} U _{i} \in \mathcal{T}_{B}\) where \(\forall j \in \left[ n \right], U_{j} \in \mathcal{T} _{B}\) by induction
            \begin{itemize}
                \item Base Case
                    \[
                    \bigcap_{i = 1}^{1} U_{i} = U_{1} \in \mathcal{T}_{B} \left(\text{by assumption}\right)
                    \]
                \item Inductive Step
                \begin{itemize}
                    \item Suppose \(k \in \mathbb{N} ^{+}\) and assume it's true for \(k\) we'll show it holds on \(k + 1\)
                        \[
                        \bigcap_{i = 1}^{k + 1} U_{i} = \bigcap_{i = 1}^{k} U_{i} \cap U_{k + 1}
                        \]
                    \item Now by induction hypothesis \(\bigcap_{i = 1}^{k} U_{i} \in \mathcal{T} _{B}\) and also we know that \(U_{k + 1} \in \mathcal{T}_{B}\), therefore by \kref{https://gitlab.com/cappajoeman/knowledge-data/-/blob/master/Intersection_of_Two_Elements_from_The_Topology_Generated_by_a_Basis_(Basis_Existance)_is_Closed-MjWZajT_hXer9YZ90uh.pdf}{the fact that this topology is closed under union for two elements}, we know that
                        \[
                        \bigcap_{i = 1}^{k} U_{i} \cap U_{k + 1} \in \mathcal{T}_{B}
                        \]
                        as needed
                \end{itemize}
            \end{itemize}
            \item We will show that for some index set \(I\) we have
                \[
                    \bigcup_{\alpha \in I} U_{\alpha} \in \mathcal{T}_{B}
                \]
            \begin{itemize}
                \item That is
                    \[
                        \forall x \in \bigcup_{\alpha \in I} U_{\alpha}, \exists B \in \mathcal{B} \text{ such that } x \in B \subseteq \bigcup_{\alpha \in I} U_{\alpha}
                    \]
                \item So let \(x \in \bigcup_{\alpha \in I} U_{\alpha}\), therefore we know that \(x \in U_{\beta}\) for some \(\beta \in I\), but since \(U _{\beta} \in \mathcal{T}_{B}\) we get \(B_{U} \in \mathcal{B}\) such that \(x \in B_{U} \subseteq U_{\beta}\)
                \item Take \(B = B_{U}\) and note that
                    \[
                    x \in B = B_{U} \subseteq U_{\beta} \subseteq \bigcup_{\alpha \in I} U _{\alpha}
                    \]
                    as needed.
            \end{itemize}
        \end{itemize}
    \end{pf}
\end{deduction*}

\end{document}


