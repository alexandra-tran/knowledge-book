\chapter*{Preface}
This book contains knowledge that that me or my peers have obtained, the purpose is to explain things fundamentally and in full detail so that someone who has never touched the subject may be able to understand it. It will focus on conveying the ideas that are involved in synthesizing the new knowledge with less of a focus on the results themselves.

It differs from a normal textbook in that it is open source and will fall under more continuous development rather than having editions that periodically come out. It also differs in the sense that it welcomes other users to improve the book.

Here is a link to the book: \url{https://github.com/cuppajoeman/knowledge-book/blob/main/build/book.pdf}


\section*{Structure of book}
The book is partitioned into different sections based on the domain it is involved with. There may be shared definitions and theorems throughout the chapters, but in general it will start more elementary and get more advanced.

\section*{Knowledge}
In this book you will find many results, will will characterize them as being one of the following
\begin{itemize}
    \item Theorems - Results that are of importantance and who's proof is not easily found (maybe using a novel idea)
  \item Propositions - Results of less importance who's proof could be constructed without a novel idea
  \item Lemmas - Results that are technical intermediate steps which has no standing as an independent result on first observation \footnote{But sometimes they escape, as their usage becomes more than just an intermediate step,  as Zorn's or Fatou's Lemmas did}
    \item Corollaries - Results which follow readily  from an existing result of greater importance
\end{itemize}

\section*{Recommendations}
By now you might know that in order to actually get better at mathematics you have to engage with it. This book may be used as a reference at times, but I highly recommend trying to re-prove statements or coming up with your own ideas before instantly looking at the solutions.

\section*{About the companion website}
The website\footnote{\url{https://github.com/cuppajoeman/knowledge-book}} for this file contains:
\begin{itemize}
  \item A link to (freely downlodable) latest version of this document.
  \item Link to download \LaTeX source for this document.
\end{itemize}

\section*{Acknowledgements}
\begin{itemize}
    \item A special word of thanks to professors who wanted to make sure I understood and learned as much as possible Alfonso Gracia-Saz\footnote{\url{https://www.math.toronto.edu/cms/alfonso-memorial/}}, Jean-Baptiste Campesato\footnote{\url{https://math.univ-angers.fr/~campesato/}}, Valentine Chiche-Lapierre and Gal Gross\footnote{\url{https://www.galgr.com/}}
    \item Thanks to Z-Module, riv, PlanckWalk, franciman, qergle from \#math on \url{https://libera.chat/}.
\end{itemize}

\chapter*{Contributing}
 
Contributions to the project are very welcome, let's delve into how to get started with this.

If you want to contribute to the project it's most likely that a contribution will fall into one of the following categories
\begin{itemize}
  \item Content Based 
  \begin{itemize}
      \item Adding Definitions, Theorems, \ldots
      \item Finishing TODO's
      \item Formatting of the book
  \end{itemize}
  \item Structural Layout of Project 
  \begin{itemize}
      \item Organization
      \item Simplyfing the existing structure of the directories 
      \item Making scripts which set up new structures
  \end{itemize}
  \item External
  \begin{itemize}
      \item Adding explanatory content to help onboard new users
      \item Getting others involved
      \item Creating infrastructure to support users (Github discussions)
  \end{itemize}
\end{itemize}

\section*{Communication}

All communication will occur through github disccusions. You can access it here: 

\section*{Content Based}

If you want to add a new top level structure, the best thing to do is to verify with other members of the project if it warrants it's own top level structure, otherwise it can be added as a substructure of an existing one.

Supposing that you are on linux, the easiest way to make a new structure and start working on it:

\begin{term}
cp -r structure new_structure
cd new_structure
mv content.tex new_structure.tex
nvim new_structure.tex
\end{term}

Otherwise if you're adding a new theorem, it could be:

\begin{term}
cd existing_structure/theorems
cp theorem.tex my_new_theorem.tex
nvim my_new_theorem.tex
...
git add -A && git commit -m "add my new theorem" && git push
\end{term}


\subsection*{Creating Files}

For example if want to create a new definition for topology we would go into the definition folder for topology and create a new file using pothole case and don't include any extraneous words, for example it is better not to append ``the" to the front of your file names when not specifically required.
