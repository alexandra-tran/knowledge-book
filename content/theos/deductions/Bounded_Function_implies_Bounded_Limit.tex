\begin{proposition}
{Bounded Function implies Bounded Limit}
    Let \(f\) be a real valued function, then and \(M \) be its global minimum, then for any \(a \in \operatorname{dom} \left(f\right)\) where the limit exists, we have
    \[
    \lim_{x \to a} f\left(x\right) \ge M
    \]
    \begin{pf}
       \begin{itemize}
           \item Let \(L = \lim_{x \to a} f\left(x\right) \) and suppose for the sake of contradiction that $ L < M $ 
           \item Consider $ \varepsilon =  M - L$, then it should be the case that 
           \[
           \exists \delta \in \mathbb{R}^{+}, \forall x \in  \operatorname{ dom }\left( f \right), \left| x - a \right| < \delta \Rightarrow \left| f\left( x \right) -  L \right| < \varepsilon 
           \]
           \item But for any $ \delta \in  \mathbb{R}^{+}  $ where $ \left| x - a \right| \le \delta  $ it cannot be true that
           \[
           \left| f\left( x \right) -  L \right| < M - L
           \]
           \item To see why we should note that 
           \[
           \left| f\left( x \right) -  L \right| \ge \left| M - L \right| =  M - L
           \]
           \item Which is clearly a contradiction.
       \end{itemize}
    \end{pf}
\end{proposition}
