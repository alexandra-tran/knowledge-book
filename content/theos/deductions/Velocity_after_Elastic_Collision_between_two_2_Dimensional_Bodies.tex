\begin{proposition}{Velocity after Elastic Collision between two 2 Dimensional Bodies}
    \begin{itemize}
        \item Given two perfectly circular bodies $ b_{1}, b_{2}$ with velocities $ \vec{v}_{1}, \vec{v}_{2}$ ,masses $ m_{1}, m_{2}$ at positions $ \vec{p}_{1}, \vec{p} _{2}$  that undergo a collision, then their new velocities are given by:
            \[
            \vec{v}_{1} =  \vec{v} _{1}    + \frac{2 m_{2}}{ m_{1}  +  m_{2}}  \left( \vec{u}_{n}  \cdot \left( \vec{v}_{2}  -  \vec{v}_{1} \right) \right) \vec{u}_{n} \quad \text{ and } \quad \vec{v}_{2} ^{\prime} =  \vec{v} _{2}    + \frac{2 m_{1}}{ m_{1}  +  m_{2}}  \left( \vec{u}_{n}  \cdot \left( \vec{v}_{1}  -  \vec{v}_{2} \right) \right) \vec{u}_{n}
            \]
    \end{itemize}
    \begin{pf}
        \subsubsection*{Setup}
        \begin{itemize}
            \item If the two bodies are colliding, then a line can be drawn that is tangent to both bodies, it passes through the single point where the two bodies are colliding, this will be denoted by the ``collision surface"
            \item With respect to the collision surface, we now have a way to consider the normal and tangential components of $ \vec{v}_{1}$ and $ \vec{v}_{2}$ 
            \begin{itemize}
                \item The normal components of the velocities undergo a one dimensional elastic collision
                \item The tangential components of the velocities are unchanged by the collision, as there is no force created along any line tangential to the collision surface from the collision
            \end{itemize}
        \end{itemize}
        \subsubsection*{Derivations}
        \begin{itemize}
            \item Variables
            \begin{itemize}
                \item A vector which is normal to the collision surface can be found by finding the vector which points from the center of one body to the other is given by:
                    \[
                    \vec{n} \stackrel{\mathtt{D}}{=} \vec{p}_{1}  -  \vec{p}_{2}
                    \]
                    and note that this is the vector which points from $ \vec{p} _{ 2}$ to $ \vec{p} _{ 1}$ 
                \begin{itemize}
                    \item To simply computations later on, let's take the vector which is in the \kref{https://gitlab.com/cuppajoeman/knowledge-data/-/blob/master/Unit_Vector_in_the_same_Direction-MjLNl9tXbkhzKlFQZYq.pdf}{same direction, but has length one}:
                        \[
                        \vec{u}_{n} \stackrel{\mathtt{D}}{=} \frac{ \vec{n}}{ \left\Vert  \vec{n} \right\Vert}
                        \]
                \end{itemize}
            \end{itemize}
            \item We will compute the new velocity of $ b_{1}$, since the tangential component of it's velocity doesn't change, we will perform some operations on $ \vec{v}_{1}$ to get to it's new velocity.
            \item To start let's zero out the component of the vector which is normal to the collision surface, to do that we will need to compute the component of $ \vec{v}_{1}$ onto the normal vector, from the  \kref{https://gitlab.com/cuppajoeman/knowledge-data/-/blob/master/Vector_Component_Formula-Mj2JQXyRzB8HZx-bUgq.pdf}{vector component formula}, thus it's equal to
                \[
                \operatorname{v-comp}_{ \vec{u}_{n}} \vec{v}_{1} = \left( \frac{ \vec{u} _{n}  \cdot  \vec{v}_{1}}{ \vec{u} _{n}  \cdot  \vec{u} _{n}} \right) \vec{u } _{n} =  \left( \vec{u}_{n}  \cdot  \vec{v}_{1} \right) \vec{u}_{n}
                \]
            \begin{itemize}
                \item Thus $\vec{v}_{1}  -  \left( \vec{u}_{n}  \cdot \vec{v}_{1} \right) \vec{u}_{n}$ represents the component of $ \vec{v}_{1}$ that is tangential to the collision surface.
            \end{itemize}
        \item We have not yet accounted for the one dimensional collision that occurs along the direction of the normal vector $ \vec{u}_{n}$, but from \kref{https://gitlab.com/cuppajoeman/knowledge-data/-/blob/master/Velocity_after_Elastic_Collision_between_two_One_Dimensional_Bodies-Mj2IJDM15JCEmh72r8f.pdf}{the velocity after a one dimensional elastic collision} we see that to bring this in. We just have to take \textcolor{blue}{the scalar} from that formula, multiply it by $ \vec{u}_{n}$ (it's a unit vector so it won't change the scalar) and add it to what we have so far, that is
            \begin{align*}
                \vec{v}_{1} ^{\prime} &=  \vec{v} _{1}  -  \left( \vec{u}_{n}  \cdot \vec{v}_{1} \right) \vec{u}_{n} + \textcolor{blue}{\left( \frac{ \left( \vec{u}_{n}  \cdot  \vec{v}_{1} \right) \left( m_{1}  -  m_{2} \right)  + 2 m_{2} \left( \vec{u}_{n}  \cdot \vec{v} _{2} \right) }{ m_{1}  +  m_{2}} \right)} \vec{u}_{n} \\
                &=  \vec{v} _{1}    + \left( \frac{ - \left( \vec{u}_{n}  \cdot \vec{v}_{1} \right) \left( m_{1}  +  m_{2} \right) +  \left( \vec{u}_{n}  \cdot  \vec{v}_{1} \right) \left( m_{1}  -  m_{2} \right)  + 2 m_{2} \left( \vec{u}_{n}  \cdot \vec{v} _{2} \right) }{ m_{1}  +  m_{2}} \right) \vec{u}_{n} \\
                &=  \vec{v} _{1}    + \left( \frac{  \left( \vec{u}_{n}  \cdot \vec{v}_{1} \right) \left( - m_{1}  -  m_{2} + m_{1}  -  m_{2} \right)   + 2 m_{2} \left( \vec{u}_{n}  \cdot \vec{v} _{2} \right) }{ m_{1}  +  m_{2}} \right) \vec{u}_{n} \\
                &=  \vec{v} _{1}    + \left( \frac{  \left( \vec{u}_{n}  \cdot \vec{v}_{1} \right) \left(  - 2 m_{2} \right)   + 2 m_{2} \left( \vec{u}_{n}  \cdot \vec{v} _{2} \right) }{ m_{1}  +  m_{2}} \right) \vec{u}_{n} \\
                &=  \vec{v} _{1}    + \left( \frac{ 2 m_{2} \left( \vec{u}_{n}  \cdot \vec{v}_{2}  -  \vec{u}_{n}  \cdot \vec{v}_{1}  \right) }{ m_{1}  +  m_{2}} \right) \vec{u}_{n} \\
                &=  \boxed{\vec{v} _{1}    + \frac{2 m_{2}}{ m_{1}  +  m_{2}}  \left( \vec{u}_{n}  \cdot \left( \vec{v}_{2}  -  \vec{v}_{1} \right) \right) \vec{u}_{n}}
            \end{align*}
        \item The setup for the second velocity is quite similar, although note that $ \vec{u}_{n}$ is pointing in the opposite direction, and thus it would invert the direction given by the one dimensional formula, thus we will flip it by multiplying it by \textcolor{red}{negative one}
            \begin{align*}
                \vec{v}_{2} ^{\prime} &=  \vec{v} _{2}  -  \left( \vec{u}_{n}  \cdot \vec{v}_{2} \right) \textcolor{red}{\left( - \vec{u}_{n} \right)} + \textcolor{blue}{\left( \frac{ \left( \vec{u}_{n}  \cdot  \vec{v}_{2} \right) \left( m_{2}  -  m_{1} \right)  + 2 m_{1} \left( \vec{u}_{n}  \cdot \vec{v} _{1} \right) }{ m_{1}  +  m_{2}} \right)} \textcolor{red}{\left(  -  \vec{u}_{n} \right)} \\
                &=  \vec{v} _{2}  +  \left( \vec{u}_{n}  \cdot \vec{v}_{2} \right) \vec{u}_{n} + \left( \frac{ \left( \vec{u}_{n}  \cdot  \vec{v}_{2} \right) \left( \textcolor{red}{m_{1}  -  m_{2}} \right)  + 2 m_{1} \left( \vec{u}_{n}  \cdot \vec{v} _{1} \right) }{ m_{1}  +  m_{2}} \right) \vec{u}_{n} \\
                &=  \vec{v} _{2}  +  \left( \vec{u}_{n}  \cdot \vec{v}_{2} \right) \vec{u}_{n} + \left( \frac{ \left( \vec{u}_{n}  \cdot  \vec{v}_{2} \right) \left( \textcolor{red}{m_{1}  -  m_{2}} \right)  + 2 m_{1} \left( \vec{u}_{n}  \cdot \vec{v} _{1} \right) }{ m_{1}  +  m_{2}} \right) \vec{u}_{n} \\
                &= \dotsc \text{similar steps as above} \\
                &=  \boxed{\vec{v} _{2}    + \frac{2 m_{1}}{ m_{1}  +  m_{2}}  \left( \vec{u}_{n}  \cdot \left( \vec{v}_{1}  -  \vec{v}_{2} \right) \right) \vec{u}_{n}}
            \end{align*}
        \end{itemize}
    \end{pf}
\end{proposition}
