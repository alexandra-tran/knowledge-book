\begin{proposition}{Little O and Differentiability Equivalence}
  A function $f$ is differentiable at the point $\overline{x}$ if and only if there is a number $ \alpha \in \mathbb{R}$ for any $h \in \mathbb{R}$ we have 
  \[
  f\left(\overline{x}  +  h\right) = f\left(\overline{x}\right)  +  h \alpha  +  l\left(h\right)
  \]
  Where $l \in o\left(h\right) $ 
  \begin{pf}
    Observe the following sequence of logic
    \begin{itemize}
      \item Definition of differentiability at $\overline{x}$, the following limit exists :
        \[
        f ^{\prime}\left(\overline{x}\right) = \lim_{h\to0}\frac{f\left(\overline{x}  +  h\right)  -  f\left(\overline{x}\right)}{h}
        \]
      \item For any $m \in \mathbb{R}$ we know that $\lim_{h\to0} m = m$, this in tandem with the sum law for limits, yields:
        \[
       \lim_{h\to0} \frac{f\left(\overline{x}  +  h\right)  -  f\left(\overline{x}\right)  -  h f ^{\prime}\left(\overline{x}\right)}{h} = 0
        \]
      \item The above is true if and only if $f\left(\overline{x}  +  h\right)  -  f\left(\overline{x}\right)  -  h f ^{\prime}\left(\overline{x}\right) \in o\left(h\right)$ 
    \end{itemize}
    Now, to the proofs
          \item Assume that $f$ is differentiable, by following the sequence of logic from top to bottom and letting $\alpha = f ^{\prime}\left(x\right)$ the proof is complete
        \end{itemize}
      \item $\Leftarrow$ 
        \begin{itemize}
          \item Assume that there is a number $\alpha \in \mathbb{R}$ and $l \in o\left(h\right)$, so that for any $h \in \mathbb{R}$  we have:
            \[
            f\left(\overline{x}  +  h\right) = f\left(\overline{x}\right)  +  h \alpha+  l\left(h\right) \Leftrightarrow l\left(h\right) = f\left(\overline{x}  +  h\right)  -  f\left(\overline{x}\right)  -  h \alpha
            \]
            thus $f\left(\overline{x}  +  h\right)  -  f\left(\overline{x}\right)  -  h \alpha \in o\left(h\right)$.
        \item Follow the logic in reverse order with $f ^{\prime}\left(\overline{x}\right)$ replaced by $\alpha$, this shows that the limit exists and equals $\alpha$, therefore the proof is concluded.
\end{proposition}

\end{document}


