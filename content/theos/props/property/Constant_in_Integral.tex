\begin{proposition}{Constant in Integral}
  For any function $f$ and constant $c \in \mathbb{R}$ 
  \[
  \int c f\left(x\right)\: dx = c \int f\left(x\right)\: dx
  \]
  \begin{pf}
    \begin{itemize}
      \item Suppose that $F\left(x\right)$ is an anti-derivative of $f\left(x\right)$, that is $F ^{\prime}\left(x\right) = f\left(x\right)$ since we can pull constants out of derivatives we have
        \[
        \left( c F\left(x\right) \right) ^{\prime} = c F ^{\prime}\left(x\right) = c f\left(x\right)
        \]
      \item Therefore $c F\left(x\right)$ is an anti-derivative of $c f\left(x\right)$, therefore
        \[
        \int c f\left(x\right)\: dx = c F\left(x\right)  +  m = c \left( F\left(x\right)  +  \frac{m}{c} \right) = c \int f\left(x\right)\: dx
        \]
    \end{itemize}
  \end{pf}
\end{proposition}
