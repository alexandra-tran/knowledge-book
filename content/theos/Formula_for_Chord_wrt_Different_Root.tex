\begin{proposition}{Formula for a Chord with respsect to a Different Root}
  Let $ \widehat{r} \mid x_{1} , x_{2} , \dotsc  , x_{n - 1} , x_{n}$ be a chord, and $ \widehat{R}$ be a different root. Then $ \widehat{r} \mid x_{1} , x_{2} , \dotsc  , x_{n - 1} , x_    {n}$ with respect to $ \widehat{R}$ is a new chord
  \[
  \widehat{R} \mid x_{1}  -  I\left( \widehat{r}, \widehat{R}\right) , x_{2}-  I\left( \widehat{r}, \widehat{R}\right) , \dotsc  , x_{n - 1}-  I\left( \widehat{r}, \widehat{R}\right) , x_{n}-  I\left( \widehat{r}, \widehat{R}\right)
  \]
  \begin{pf}
    \begin{itemize}
      \item We know that there exists $k \in \mathbb{Z}$ such that $ \widehat{r}  +  k = \widehat{R}$, namely $k = I\left( \widehat{r}, \widehat{R}\right)$                               
      \item We want to find out what the interval is from $ \widehat{R}$ to  $x_{i}$ , written as $I\left( \widehat{R}, \widehat{r  +  x_{i}}\right)$ which is defined to be equal to: $ r  +  x_{i}  -  R$
        \begin{itemize}
          \item But we already know that $R = r  + k$, therefore that equation becomes
      \item Since $k = I\left( \widehat{r}, \widehat{R}\right)$, then we can conclude that 
        \[
        I\left( \widehat{R}, \widehat{r  +  x_{i}}\right)= x_{i}  -  I\left( \widehat{r}, \widehat{R}\right)
        \]
      \item That is, each of the intervals from the first chord get transformed to intervals in the second chord by subtracting the interval from $ \widehat{r}$ to $ \widehat{R}$ 
    \item Say there is a change from $ \widehat{8} \mid 0, 3, 7, 9,$ (Ab 6) to $ \widehat{1} \mid 0, 4, 7, 10$ (Db 7)
      \begin{itemize}
        \item Using the above Idea we can see that $I\left( \widehat{8}, \widehat{1}\right) = 5$ and so we we can subtract 5 from each interval and take it modulo 12 to get our new intervals with respect to the new root:
    \item The upshot is that now we can easily see how the chord we are playing now can be tweaked to fit the next chord's context.
  \end{itemize}
\end{proposition}
