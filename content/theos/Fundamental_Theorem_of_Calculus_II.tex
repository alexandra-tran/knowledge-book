\begin{proposition}{Fundamental Theorem of Calculus II}
  If $f$ is continuous on $ \left[ a, b \right]$ then:
      \item We can see that $h$ is continuous on $ \left[ a, b \right]$ and differentiable on $ \left(  a, b \right)$ as it is a difference of two functions with those same properites. Further, for any $x \in  \left( a, b \right)$ we get
        \[
        h ^{\prime}\left(x\right) = g ^{\prime}\left(x\right)  -  F ^{\prime}\left(x\right)
        \]
      \item By FTC part I, and the fact that $F ^{\prime}\left(x\right) = f\left(x\right)$ by definition of $F$ we obtain that 
        this in tandem with the fact that $h$ is continuous satifies the hypothesis for $h$ to be constant on $ \left[ a,b \right]$ therefore $h\left(a\right) = h\left(b\right)$ 
      \item This yields the following
        \begin{gather*}
          h\left(b\right) = h\left(a\right) \\
          \Updownarrow \text{(Def.)} \\
          g\left(b\right)  -  F\left(b\right) = g\left(a\right)  -  F\left(a\right)\\
          \Updownarrow \text{(Alg.)} \\
          g\left(b\right)  =    g\left(a\right) + F\left(b\right)  -  F\left(a\right)\\
          \Updownarrow \text{(Def.)} \\
          \int _{a} ^{b} f\left(t\right)\: dt = \int _{a} ^{a} f\left(t\right)\: dt   +  \left( F\left(b\right)  -  F\left(a\right) \right) \\
          \Updownarrow \text{(Integral Property)} \\
          \int _{a} ^{b} f\left(t\right)\: dt = 0  +  F\left(b\right)  -  F\left(a\right) \\
          \Updownarrow \text{(Simp.)} \\
          \int _{a} ^{b} f\left(t\right)\: dt = F\left(b\right)  -  F\left(a\right)
        \end{gather*}
\end{proposition}

\end{document}


