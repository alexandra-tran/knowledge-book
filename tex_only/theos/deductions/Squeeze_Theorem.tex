\begin{deduction*}{Squeeze Theorem}
    Suppose that $ \lim_{ x \to a } g\left( x \right) =  \lim_{ x \to a } h\left( x \right) = L $ and for some interval $ I \setminus \left\{ a \right\} $ we have that $ g\left( x \right) \le f\left( x \right) \le h\left( x \right) $  , then the following holds: 
    \[
        \lim_{ x \to a } f\left( x \right) = L
    \]
    \begin{pf}
        \begin{itemize}
            \item Let $ \varepsilon \in \mathbb{R}^{+}  $, we must find a $ \delta \in  \mathbb{R}^{+}  $ where for any $ x \in \operatorname{dom} \left( f \right)   $, where $ 0 < \left| x -  a \right| < \delta   $ 
                \[
                \left| f\left( x \right) - L \right| < \varepsilon \Leftrightarrow -  \varepsilon < f\left( x \right)  - L < \varepsilon 
                \]
            \item From the fact that both of the limits of $ g, h $ exist we know that there exists $ \delta _{ g } , \delta _{ h }  $ for our $ \varepsilon  $  such that in their respective punctured intervals we have that
                \[
                - \varepsilon < g\left( x \right) - L < \varepsilon  \quad \text{ and } \quad - \varepsilon < h\left( x \right) -  L < \varepsilon 
                \]
            \item Now we can see that
                \begin{gather*}
                    g\left( x \right)  \le f\left( x \right)  \le h\left( x \right) \\
                    \Updownarrow \\
                    g\left( x \right) - L  \le f\left( x \right) - L  \le h\left( x \right) - L \\
                    \Updownarrow \\
                    - \varepsilon  < g\left( x \right) - L  \le f\left( x \right) - L  \le h\left( x \right) - L < \varepsilon  \\ 
                    \Updownarrow \\
                    - \varepsilon  <  f\left( x \right) - L  < \varepsilon 
                \end{gather*}
            \item As needed.
        \end{itemize}
    \end{pf}
\end{deduction*}
