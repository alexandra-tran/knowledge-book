\begin{deduction*}{GCD Invariant for Remainder}
  Let $ m,n \in \mathbb{N} ^{\ge 1}$ such that $ m > n$ and let $ q$ and $ r$ be the unique integers from the quotient remainder theorem, that is, they satisfy
  \[
    m =  q  \cdot n  +  r \qquad \text{ and } \qquad 0 \le r < n
  \]
  Then if $ r > 0$ we have:
  \[
  \gcd(m,n) = \gcd(n,r)
  \]
  or if $ r = 0$, then $ n \mid m$ and $ \gcd(m,n)= n$ 
  \begin{pf}
    \begin{itemize}
      \item If $ r > 0$ 
        \begin{itemize}
          \item Then the equivalent formulation:
            \[
            m - q  \cdot n =  r
            \]
          shows us that if $ d \mid m$ and $ d \mid n$ then $ d \mid r$
          \item Going the other direction would be assuming that $ d$ is a divisor of $ n$ and $ r$, and showing that $ d$ also divides $ m$. This is clear from the original equation:
            \[
              m = q  \cdot  n  +  r
            \]
          \item Therefore if we consider all the divisors of $n$ and $m$ each one of them also divides $r$, so the set of divisors of $m$ and $n$ is  
            \[
              \mathcal{D}=\left\{ d \in \mathbb{N}: d \mid n \land d \mid m \land d \mid r \right\}
            \]
          \item We also consider all the divisors $d$ of $n$ and $r$, from our previous observations $d$ also divides $m$ so the set of divisors is also $\mathcal{D}$, therefore the maximum element from both of these sets is the same and we have :
            \[
              \gcd(n,m)=\gcd(n,r)
            \]

        \end{itemize}
        \item If $ r =  0$ then we know that $ m = q  \cdot n$ which is the definition of $ n \mid m$ and it's clear that $ \gcd(m, n) = \gcd(q  \cdot n, n) =  n$ 
    \end{itemize}
  \end{pf}
\end{deduction*}
