\begin{defn*}{Factorial}
  \subsubsection*{Direct}
  Let $n \in \mathbb{N}$, then 
  \[
  n! \stackrel{\mathtt{D}}{=} n  \cdot  \left( n -1  \right)  \cdot  \left( n - 3 \right) \dotsb 3  \cdot 2  \cdot 1
  \]
  \subsubsection*{Recursive}
  \[
    n! = 
    \begin{cases}
      1 \text{ if } n = 0\\
      n \cdot \left( n - 1 \right)! \text{ otherwise }
    \end{cases}
  \]

  \subsubsection*{0! = 1}
  \begin{itemize}
    \item The definition of $n!$ as a product of no numbers at all is an example of the convention that the product of no factors is equal to the multiplicative identity 
    \item You can see that $\frac{x!}{x} = \left( x-1 \right)!$, so $\frac{1!}{1} = 0!$ but $\frac{1!}{1} = 1$ so it makes sense for $0! = 1$ 
  \end{itemize}
\end{defn*}
